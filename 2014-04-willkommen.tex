\documentclass[a4paper]{letter}
\usepackage{../dotfiles-etc/mystyle-letter}

\name{Verena Kasztantowicz \& Maximilian Held}
\address{
	Verena Kasztantowicz \\
	Kastanienallee 11 \\
	13129 Berlin \\
	verenakasztantowicz@gmail.com \\
	+49.151.26959663\\
	\\
	Maximilian Held \\
	Kiefernweg 35 \\
	28857 Syke \\
	maximilian.held83@gmail.com \\
	+49.151.22958775
	}

\date{15.\ April 2014}

\location{Bremen und Berlin}

\begin{document}

\begin{letter}{
	Teilnehmerinnen und Teilnehmer des Kurses 4.4 \\
	Deutsche SchülerAkademie 2014-2 in Braunschweig}

\opening{Liebe Teilnehmerinnen und Teilnehmer,}

Herzlich Willkommen im Kurs 4.4 der Deutschen SchülerAkademie 2014 ``Schule der Demokratie -- deliberative Demokratie und inklusive Pädagogik''!

Wir sind Verena und Max, Deine Kursleiter bei der Deutschen SchülerAkademie in diesem Sommer.
Wir freuen uns, dass Du dich für unseren Kurs ``Schule der Demokratie'' angemeldet hast und versprechen Dir, dass Du deine Wahl nicht bereuen wirst.
Wir sind gespannt auf die gemeinsame Arbeit im Kurs und sind uns sicher, dass wir während der Akademie in Braunschweig jede Menge bewegen werden können -– in unserem Kursraum, in unserer Wissenschaft und in unseren Köpfen.
% Wir beide sind schon seit vielen Jahren bei der SchülerAkademie dabei und waren beide auch einmal an Eurer Stelle, als Teilnehmer.
% Seitdem gehört SchülerAkademie für uns zu einem guten Sommer dazu.
% Dieses Jahr: mit Dir.

% Jetzt wollen wir erstmal Deine Neugier auf unser Thema wecken und nebenher ein paar technische Details klären.

% Ein paar Allgemeine Sätze zu Thema und Ziel des Kurses.
% Wie in der Kursbeschreibung bereits geschildert, wollen wir in unserem Kurs eine Spanne schlagen zwischen zwei Fächern, die in der wissenschaftlichen Praxis oft nicht viel Austausch pflegen: der Soziologie (genauer der politischen Ökonomie, die Max vertritt) und der Philosophie (genauer der Ideengeschichte, die Jonas vertritt). Wir warnen gleich: wir werden keine der beiden Disziplinen in vollem Umfang vorstellen können. “ARBEIT” ist aber ein so umfassendes und komplexes Thema, dass uns eine Wissenschaft nicht genügt, um darüber nachzudenken. Deshalb wollen wir konkrete Fragen - auch aus dem Alltag - mit Ideen und Abstraktionen beider Wissenschaften diskutieren. Wir sind selber gespannt, was aus diesem schizodisziplinären Abenteuer wird.
% Auch die Idee zum Thema des Kurses entstammt nicht dem gängigen Kanon einer der beiden Disziplinen, sondern unserem persönlichen Interesse. Wir glauben, dass die Bedingungen und Möglichkeiten menschlicher Arbeit nachhaltig Einfluss darauf haben, wie wir als Gemeinschaft miteinander leben und umgehen. Und wir glauben, dass zu diesem Thema noch lange nicht alles gesagt ist.
% Zunächst wollen wir uns im Kurs die (ökonomischen) Abstraktionen und (philosophischen) Ideen zur Arbeit erschließen und von dort aus auch allgemeine Fragen über das Verhältnis von Mensch und Welt skizzieren. Es geht aber nicht nur um Theorie: Arbeit prägt Lebenschancen und Lebenszeit von Menschen. Wir haben es hier also mit einem eminent politischen Thema zu tun und dementsprechend
% 
% wollen wir - aufbauend auf unseren intellektuellen Überlegungen - auch über konkrete Reformvorschläge diskutieren.
% Absolute Wahrheiten und einfache Antworten wird es dabei in unserem Kurs nicht geben. Daher wird unser Kurs auch nicht im Stile einer frontalen Vermittlung von abstraktem Wissen ablaufen (auch wenn das an der einen oder anderen Stelle mal nötig sein wird). Wir stellen uns ein Lernen im Gespräch und in der gemeinsamen Diskussion vor. Und wir wollen, dass Du auch Deine eigenen Erfahrungen und Fragen, die sich Dir daraus ergeben, einbringen kannst und sollst. Manchmal müssen wir dabei aber auch gängige Denkmuster und Überzeugung darüber, was Arbeit sei oder sein sollte, kritisch hinterfragen - vielleicht sogar erst einmal ganz vergessen -, um uns auf wissenschaftliche Abstraktionen und Ideen einlassen zu können.
% Die Vorbereitung
% Der aufwändigere Teil der Vorbereitung beginnt später.
% Wie in allen Kursen werden auch wir Anfang bis Mitte Juni einen Reader mit den wichtigsten Texten verschicken und Dich bitten, diese bis zum Beginn der Akademie gründlich vorzubereiten. Zu einigen der Texte solltest Du (evtl. mit anderen Teilnehmenden in Kleingruppen) ein kurzes Referat und Diskussionsfragen vorbereiten. Es wäre also gut, wenn Du in den Wochen vor der Akademie etwas Zeit für diese Aufgaben einplanen kannst. Mehr dazu, auch zur Referatsvergabe, werden wir Dir mit der Reader-Verschickung Anfang bis Mitte Juni mitteilen.
% Erster Lesestoff
% Bis dahin möchten wir Dir ein Buch empfehlen:
% Schwartz, B., & Sharpe, K. (2010). Practical Wisdom - The Right Way to Do the Right Thing. New York, NY: Penguin Books.
% ``
% Practical Wisdom'' ist ein Buch, das sich nicht direkt mit dem Thema „Arbeit“ beschäftigt, aber allgemein mit der Frage, wie und nach welchen Maximen wir im Alltag miteinander umgehen sollten. Dabei spielen Beispiele aus der Arbeitswelt immer wieder eine wichtige Rolle, vor allen Dingen beziehen sich die Autoren aber ausführlich auf antike Überzeugungen, die sie bei Aristoteles finden und die auch uns im Kurs näher beschäftigen werden. Die Frage nach den Möglichkeiten einer „praktischen Weisheit“ werden auch uns im Kurs beschäftigen, wenn wir verschiedene Konzepte und Formen des Arbeitens wissenschaftlich untersuchen.
% Wir können Dir das Buch dementsprechend sehr empfehlen - im Gegensatz zum Reader ist es aber keine Pflichtlektüre. Du wirst im Kursverlauf keinen Nachteil haben, wenn Du es nicht liest. Das Buch ist unterhaltsam geschrieben und leicht zu lesen, liegt aber nur in englischer Sprache und leider nur als Hardcover (EUR 19,99) vor. Vielleicht magst Du das Buch kaufen, wir haben es aber auch als gescanntes PDF zum Download bereit gestellt:
% http://dl.dropbox.com/u/5341489/Practical_Wisdom.pdf
% Wenn Du erst einmal einen kleinen Vorgeschmack haben möchtest, kannst Du auch einen etwa 20- minütigen Vortrag des Autors zu dem Thema anschauen. Gib einfach „Barry Schwartz Practical Wisdom“ bei Youtube ein, dann sollte es der erste Treffer sein, oder nimm alternativ den direkten Link:￿
% http://www.youtube.com/watch?v=IDS-ieLCmS4
% Die Technik
% In Vorbereitung und Durchführung einer Akademie gibt es eine ganze Menge an Informationen und Texten zu verarbeiten. Wir wollen das so effizient und elegant wie möglich tun, damit mehr Zeit und Lust für inhaltliche Arbeit bleibt. Deshalb hier ein paar technische Details.
% Wir wollen alle Kommunikationen im Vorfeld der Akademie so weit möglich auf zwei Kanälen führen, per E-Mail und in einem Wiki. Per E-Mail kannst Du uns kontaktieren und wir können Dir alle wichtigen Infos zukommen lassen. Im Wiki könnt Ihr Euch gegenseitig im Vorfeld austauschen, insbesondere auch bei Rückfragen zum Verständnis der Reader-Texte.
% Um all dies in Gang bringen zu können, brauchen wir von Dir jetzt erst einmal nur eine aktuelle E- Mail-Adresse und möchten Dich bitten, uns kurz an beide oben angegebene Adressen eine Rückmeldung mit der aktuellen E-Mail-Adresse zu geben. Diese Adresse wird dann auch Dein Zugang zum Wiki, dass wir spätestens mit der Verschickung des Readers online schalten wollen. Wenn Du eine Google E-Mail-Adresse hast, oder planst, eine anzulegen, schick’ uns am besten die. Jede andere geht aber auch.
% Bitte rufe Deine E-Mails regelmäßig ab und stell' sicher, dass wir nicht in Deinem SPAM-Ordner landen (wir sprechen hier aus eigener, schmerzhafter Erfahrung). Bitte lass' uns wissen, falls Du keine E-Mail oder keinen regelmäßigen Internetzugang einrichten kannst, dann lassen wir uns etwas einfallen. Bitte benachrichtige uns auch wenn sich deine E-Mail Adresse ändern sollte.
% Kontaktiere Uns
% Wenn Du Fragen hast, schreib' uns einfach. Am besten sind wir per E-Mail zu erreichen, wenn es dringendes zu klären gibt, kannst Du uns aber auch gerne anrufen. Die Kontaktdaten stehen am Anfang dieses Briefes. Wir freuen uns, von Dir zu hören.
% Bis im Sommer!
% Wir sind auf jeden Fall schon gespannt auf Dich, auf Deine Ideen und Beiträge, auf die Gruppe insgesamt, auf unsere gemeinsame Arbeit und Ergebnisse und natürlich auch auf die schönen Momente außerhalb des Seminarraums mit den anderen Teilnehmenden!
% In Vorfreude auf die gemeinsame Zeit im Sommer,
% Herzliche Grüße
% Deine Kursleiter Jonas & Max

% Der Reader zum Kurs ``ARBEIT - Ideengeschichte und politische Ökonomie''
% Liebe Kursteilnehmerin, Lieber Kursteilnehmer,
% jetzt geht es endlich los. Leider etwas verspätet findet ihr hier in unseren Reader für die gemeinsame Arbeit im Sommer. Ausserdem erhaltet ihr in diesen Tagen Zugang zu unserem Kurswiki:
% https://sites.google.com/site/arbeit2011dsa
% Bitte arbeitet den Reader bis zum Beginn der Akademie sorgfältig durch. Wir haben dafür im Wiki einige Aufgaben eingestellt.
% Zunächst bitten wir euch, alle Texte im Reader zu Lesen und die für jeden Text eingestellten Lesefragen im Wiki zu beantworten und zu kommentieren. Die Lesefragen dienen Euch als Hilfe zum Textverständnis, sowie als Anregung und Forum für eine lebhafte Diskussion unter Euch auch schon vor dem eigentlichen Beginn der Akademie. Uns geht es uns￿nicht￿darum, Euer Verständnis oder Eure Leistung zu￿testen, wir sind ja nicht in der Schule, sondern auf der SchülerAkademie.￿
% Außerdem bitten wir auch, jeweils mit einer/m anderen Kursteilnehmer/in zum einen ein Referatsthema vorzubereiten,￿und￿und zum anderen Schwerpunkte für Diskussion für ein anderes der Referatsthemen zu überlegen. Jeder von Euch wird also sowohl jeweils einmal "Referent" und einmal "Diskutant" sein. Damit das möglichst einfach und ohne großes E-Mail- Chaos geht, bitten wir Euch, Euch selbstständig in der im Wiki eingebetteten Tabelle für die Aufgaben einzutragen. Bitte sprecht Euch sorgfältig untereinander ab, damit jeder ein Thema bearbeiten kann, was ihr/ihm gefällt (nicht: "wer zuerst kommt, mahlt zuerst..."). Tragt euch bitte verbindlich bis zum 15. Juli für die Aufgaben ein.
% Die Referate verstehen sich als Einstieg in das jeweilige Thema und sollen prägnant die wichtigsten Thesen zu einem Thema präsentieren ohne allzu sehr ins Detail zu gehen (das machen wir dann in der gemeinsamen Diskussion). Die Referate sollen dementsprechend nicht
% länger als 10 Minuten dauern. Als Diskutant sollt ihr Euch darauf vorbereiten, mit prägnanten Thesen und Fragen die Diskussion zum jeweiligen Thema zu unterfüttern.
% Jetzt wünschen wir Euch viel Vergnügen und Durchhaltevermögen beim Durchlesen des Readers. Wir versprechen Euch, (die) Arbeit ist es wert.
% Schaut Euch auch im Wiki um. Wir haben da einige Lesetipps und andere Anregungen zusammen getragen. Es gibt im Wiki auch einen Blog, in dem wir uns schon vor der Akademie über alles mögliche austauschen können.￿
% Natürlich könnt ihr uns auch weiterhin jederzeit eine E-Mail schreiben oder anrufen, wenn ihr eine Frage habt.
% Bis in fünf Wochen!
% Herzliche Grüße
% Deine Kursleiter Jonas & Max


% Der Reader zum Kurs ``ARBEIT - Ideengeschichte und politische Ökonomie''
% Liebe Kursteilnehmerin, Lieber Kursteilnehmer,
% jetzt geht es endlich los. Leider etwas verspätet findet ihr hier in unseren Reader für die gemeinsame Arbeit im Sommer. Ausserdem erhaltet ihr in diesen Tagen Zugang zu unserem Kurswiki:
% https://sites.google.com/site/arbeit2011dsa
% Bitte arbeitet den Reader bis zum Beginn der Akademie sorgfältig durch. Wir haben dafür im Wiki einige Aufgaben eingestellt.
% Zunächst bitten wir euch, alle Texte im Reader zu Lesen und die für jeden Text eingestellten Lesefragen im Wiki zu beantworten und zu kommentieren. Die Lesefragen dienen Euch als Hilfe zum Textverständnis, sowie als Anregung und Forum für eine lebhafte Diskussion unter Euch auch schon vor dem eigentlichen Beginn der Akademie. Uns geht es uns￿nicht￿darum, Euer Verständnis oder Eure Leistung zu￿testen, wir sind ja nicht in der Schule, sondern auf der SchülerAkademie.￿
% Außerdem bitten wir auch, jeweils mit einer/m anderen Kursteilnehmer/in zum einen ein Referatsthema vorzubereiten,und und zum anderen Schwerpunkte für Diskussion für ein anderes der Referatsthemen zu überlegen. Jeder von Euch wird also sowohl jeweils einmal "Referent" und einmal "Diskutant" sein. Damit das möglichst einfach und ohne großes E-Mail- Chaos geht, bitten wir Euch, Euch selbstständig in der im Wiki eingebetteten Tabelle für die Aufgaben einzutragen. Bitte sprecht Euch sorgfältig untereinander ab, damit jeder ein Thema bearbeiten kann, was ihr/ihm gefällt (nicht: "wer zuerst kommt, mahlt zuerst..."). Tragt euch bitte verbindlich bis zum 15. Juli für die Aufgaben ein.
% Die Referate verstehen sich als Einstieg in das jeweilige Thema und sollen prägnant die wichtigsten Thesen zu einem Thema präsentieren ohne allzu sehr ins Detail zu gehen (das machen wir dann in der gemeinsamen Diskussion). Die Referate sollen dementsprechend nicht
% länger als 10 Minuten dauern. Als Diskutant sollt ihr Euch darauf vorbereiten, mit prägnanten Thesen und Fragen die Diskussion zum jeweiligen Thema zu unterfüttern.
% Jetzt wünschen wir Euch viel Vergnügen und Durchhaltevermögen beim Durchlesen des Readers. Wir versprechen Euch, (die) Arbeit ist es wert.
% Schaut Euch auch im Wiki um. Wir haben da einige Lesetipps und andere Anregungen zusammen getragen. Es gibt im Wiki auch einen Blog, in dem wir uns schon vor der Akademie über alles mögliche austauschen können.
% Natürlich könnt ihr uns auch weiterhin jederzeit eine E-Mail schreiben oder anrufen, wenn ihr eine Frage habt.
% Bis in fünf Wochen!
% Herzliche Grüße
% Deine Kursleiter Jonas & Max

\end{letter}

\end{document}