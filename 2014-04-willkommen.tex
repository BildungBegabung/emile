\documentclass[a4paper]{letter}
\usepackage{../dotfiles-etc/mystyle-letter}

\name{Verena \& Max}
\address{
	Verena Kasztantowicz \\
	Kastanienallee 11 \\
	13129 Berlin \\
	\href{mailto:verenakasztantowicz@gmail.com}{verenakasztantowicz@gmail.com}  \\
	+49.151.26959663\\
	\\
	Maximilian Held \\
	Kiefernweg 35 \\
	28857 Syke \\
	\href{mailto:maximilian.held83@gmail.com}{maximilian.held83@gmail.com}  \\
	\url{http://www.maxheld.de} \\
	+49.151.22958775
	}

\date{15.\ April 2014}

\location{Bremen und Berlin}

\begin{document}

\begin{letter}{
	Teilnehmerinnen und Teilnehmer des Kurses 2.4 \\
	Deutsche SchülerAkademie 2014-2 in Braunschweig}

\opening{Liebe Teilnehmerinnen und Teilnehmer,}

Herzlich Willkommen im Kurs 2.4 der Deutschen SchülerAkademie 2014 "`Schule der Demokratie --- Deliberative Demokratie und Inklusive Pädagogik"'!

Wir sind Verena und Max, deine Kursleiter bei der Deutschen SchülerAkademie in diesem Sommer.
Wir freuen uns, dass du dich für unseren Kurs "`Schule der Demokratie"' angemeldet hast und versprechen dir, dass du deine Wahl nicht bereuen wirst.
Wir sind gespannt auf die gemeinsame Arbeit im Kurs und sind uns sicher, dass wir während der Akademie in Braunschweig jede Menge bewegen werden --– in unserem Kursraum, in unserer Wissenschaft und in unseren Köpfen.

Wie immer bei der Deutschen SchülerAkademie gibt es auch für unseren Kurs einiges vorzubereiten, bevor es richtig losgehen kann.
Du bekommst von uns in ein paar Wochen verschiedene Texte (etwa 80 Seiten), die wir dich bitten sorgfältig durchzuarbeiten.
Einen der Texte wirst du auch in einer Präsentation den anderen Kursteilnehmerinnen und Kursteilnehmern vorstellen.
Außerdem haben wir noch ein paar Aufgaben vorbereitet, die dir den Zugang zu den Texten erleichtern sollen.
Plane deshalb am besten in den Wochen vor der Akademie etwas Zeit für die Vorbereitung ein.

Wir sind natürlich auch neugierig darauf dich kennenzulernen!
Bitte schicke uns deshalb doch schon mal eine E-Mail, in der du uns ganz kurz beschreibst, was dich zu deiner Kurswahl bewegt hat und welche Fragen und Erwartungen du mitbringst.
Vielleicht hast du auch schon einen Vorschlag, was wir unbedingt im Kurs behandeln sollten?
Wie angekündigt, werden wir für und auf der Akademie viel digital arbeiten, es wäre also wichtig, dass du uns von einer E-Mail-Adresse schreibst, die zuverlässig funktioniert und die du regelmäßig abrufst.
Wir werden dich dann zukünftig nur noch unter dieser Adresse kontaktieren (z.B. für die Texte).
Bitte lass uns wissen, falls du keine E-Mail oder keinen regelmäßigen Internetzugang einrichten kannst, dann lassen wir uns etwas einfallen.

Um deine Neugierde auf unseren gemeinsamen Kurs zu wecken, haben wir darüber nachgedacht, was dich denn so alles erwarten wird in diesen drei Wochen.
Weil unser Begrüßungsbrief damit aber 80 Seiten Text bereits maßlos überschreiten würde, beschränken wir uns lieber darauf, was unser Kurs alles \emph{nicht} ist:
Auch wenn der Titel "`Schule der Demokratie"' es vermuten lässt, geht es in diesem Kurs \emph{nicht} darum, euch und uns zu demokratischeren Menschen zu erziehen --- wenn überhaupt, dann ist die Frage eher umgekehrt: Wie kann Schule, wie können Lernen und letztlich auch Beziehungen demokratischer gestaltet werden?
Schließlich geht es in unserem Kurs auch \emph{nicht} um das, was man gemeinhin unter Politik versteht --- vor allem keine Tages- oder Schulpolitik --- sondern allgemeiner darum, was eigentlich kollektiv verbindlichen Entscheidungen Legitimität verleihen könnte.

Wir arbeiten beide schon seit ein paar Jahren an unseren jeweiligen Themen: Verena an Inklusiver Pädagogik, also dem gemeinsamen Lernen von Menschen mit verschiedenen Voraussetzungen und Max an Deliberativer Demokratie, also der gemeinsamen Entscheidungsfindung durch den Austausch von Begründungen.
In unseren Gesprächen ist uns deutlich geworden, dass diese zwei Perspektiven einige gemeinsame Fragen aufwerfen.
Eine dieser Fragen ist:
Wie können unterschiedliche und unterschiedlich \emph{fähige} Menschen zusammenleben und entscheiden, und dabei persönlich selbständig und inhärent gleich(wertig) bleiben, oder werden?

Zugegeben, das ist eine ziemlich große und abstrakte Frage für den Anfang, selbst wenn wir "`nur"' Antworten für einen kleinen Ausschnitt des Ganzen suchen, nämlich für Schule und Demokratie.
Doch wir vertrauen darauf, gemeinsam im Kurs solche abstrakten Gedanken nachverfolgen zu können und uns dabei auf schwierige Texte einzulassen, die zunächst scheinbar wenig mit fassbaren, praktischen Dingen zu tun haben.

Zum einen denken wir, kann Wissenschaft als Geistessport Freude bereiten, besonders, wenn man, wie wir in "`Schule der Demokratie"', einen Überblick über verschiedene Zeiten und Disziplinen akademischen Wirkens bekommt.
Zum anderen sind Theorie und Praxis \emph{durchaus} gleich --- jedenfalls theoretisch, wie der amerikanische Baseball-Trainer Yogi Berra einmal anmerkte.
Wer --- wie vielleicht ihr und wir --- ganz praktisch etwas verändern will, der muss erstmal wissen, \emph{was} denn verändert werden soll, und \emph{warum}, d.h. gute Gründe dafür finden.
Dies gilt besonders für "`Schule der Demokratie"', in dem wir uns schließlich mit zwei einigermaßen radikal-progressiven Reformvorschlägen befassen, die durchaus kritisch diskutiert werden: Inklusive Pädagogik und Deliberative Politik.

Das sorgfältiges Nachdenken über Abstraktionen, das Hadern mit Begriffen und das Verständnis von anderer Leute Ideen ermöglicht es manchmal erst, Forderungen zu klären, und, vor allem, \emph{neue} Möglichkeiten zu entdecken.

Bis in ein paar Wochen die intensivere Kursvorbereitung beginnt, möchten wir dir das 2008 im DTV erschienene Buch "`Warum muss ich zur Schule gehen?"' von Hartmut von Hentig empfehlen.
Von Hentig wurde einmal von seinem kleinen Neffen Tobias gefragt: "`Warum muss ich eigentlich zur Schule gehen?"' (was dem einen oder anderen an einem Montagmorgen nach den Osterferien vielleicht nicht ganz unbekannt vorkommen wird.)
Glück für Tobias, dass sein Onkel einer der berühmtesten deutschen Pädagogen des 20. Jahrhunderts ist und die Frage damit nicht unbeantwortet lassen konnte.
In 26 Briefen versucht er, Tobias Antworten auf seine Frage zu geben und schlägt dabei --- ganz ähnlich wie unser Kurs --- einen weiten Bogen von Politik, Pädagogik, Philosophie, Psychologie bis hin zu ganz persönlichen Erfahrungen.
Aus einer scheinbar banalen Kinderfrage entwickelt von Hentig tiefgreifende Gedankenanstöße und stellt die These auf, dass Demokratie im Großen immer erst im Kleinen anfangen muss.

Das Buch scheint momentan vergriffen, ist aber im Internet (etwa: \url{http://www.booklooker.de}) günstig gebraucht zu kaufen.
Wir stellen es in den nächsten Tagen auch als PDF bereit, und schicken dir den Link als Antwort per E-Mail.
Der Text sei Dir als Einstieg in das Thema empfohlen, die Lektüre ist aber  keine Verpflichtung.

Wenn du Fragen hast, schreib' uns einfach.
Am besten sind wir per Mail zu erreichen, wenn es dringendes zu klären gibt, kannst du uns aber auch gerne anrufen. Die Kontaktdaten stehen am Anfang dieses Briefes.

Wir freuen uns auf dich, auf deine Ideen und Beiträge, auf die Gruppe, auf unsere gemeinsame Arbeit und Ergebnisse und natürlich auch auf die schönen Momente außerhalb des Seminarraums mit den anderen Teilnehmenden!

Mit Vorfreude auf die gemeinsame Zeit im Sommer und bis bald!

\closing{

	Deine Kursleiter
}

\end{letter}

\end{document}
