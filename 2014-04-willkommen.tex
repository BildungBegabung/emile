\documentclass[a4paper]{letter}
\usepackage{../dotfiles-etc/mystyle-letter}

\name{Verena \& Max}
\address{
	Verena Kasztantowicz \\
	Kastanienallee 11 \\
	13129 Berlin \\
	verenakasztantowicz@gmail.com \\
	+49.151.26959663\\
	\\
	Maximilian Held \\
	Kiefernweg 35 \\
	28857 Syke \\
	maximilian.held83@gmail.com \\
	+49.151.22958775
	}

\date{15.\ April 2014}

\location{Bremen und Berlin}

\begin{document}

\begin{letter}{
	Teilnehmerinnen und Teilnehmer des Kurses 4.4 \\
	Deutsche SchülerAkademie 2014-2 in Braunschweig}

\opening{Liebe Teilnehmerinnen und Teilnehmer,}

Herzlich Willkommen im Kurs 4.4 der Deutschen SchülerAkademie 2014 ``Schule der Demokratie -- deliberative Demokratie und inklusive Pädagogik''!

Wir sind Verena und Max, Deine Kursleiter bei der Deutschen SchülerAkademie in diesem Sommer.
Wir freuen uns, dass Du dich für unseren Kurs ``Schule der Demokratie'' angemeldet hast und versprechen Dir, dass Du deine Wahl nicht bereuen wirst.
Wir sind gespannt auf die gemeinsame Arbeit im Kurs und sind uns sicher, dass wir während der Akademie in Braunschweig jede Menge bewegen werden können -– in unserem Kursraum, in unserer Wissenschaft und in unseren Köpfen.

Wie immer der Deutschen SchülerAkademie gibt es auch für unseren Kurs einiges vorzubereiten.
Ihr bekommt von uns in ein paar Wochen verschiedene Texte (etwa 80 Seiten), die wir Dich bitten sorgfältig durchzuarbeiten.
Einen der Texte wirst Du dann auch in einer Präsentation den anderen Kursteilnehmerinnen und Kursteilnehmern vorstellen.
Außerdem haben wir noch einige kleinere Aufgaben vor, die Dir den Zugang zu den Texten erleichtern werden.
Plane deshalb am besten in den Wochen vor der Akademie etwas Zeit für die Vorbereitung ein.
Alles Weitere dazu in ein paar Wochen.

Wir sind aber natürlich auch schon neugierig Dich etwas kennen zu lernen.
Bitte schicke uns deshalb doch schonmal eine E-Mail, in der Du uns kurz beschreibst was Dich zu deiner Kurswahl bewegt hat, und welche Fragen und Erwartungen Du mitbringst.
Vielleicht hast Du auch schon einen Vorschlag, was wir unbedingt im Kurs behandeln sollten?
Wir werden Dich dann zukünftig auch unter dieser E-Mail Adresse kontaktieren.
Wie angekündigt werden wir für und auf der Akademie viel digital arbeiten, es wäre also wichtig, dass Du uns eine Adresse von Dir gibst, die zuverlässig funktioniert und die Du regelmäßig abrufst.
Bitte lass' uns wissen, falls Du keine E-Mail oder keinen regelmäßigen Internetzugang einrichten kannst, dann lassen wir uns etwas einfallen.

Heute wollen wir erstmal Deine Neugier auf unser gemeinsames Thema wecken, und ein paar Fragen klären, die sich Dir vielleicht schon stellen.
Zunächst ein paar Dinge, die unser Kurs alles \emph{nicht} ist.
Auch wenn der Titel ``Schule der Demokratie'' es vermuten lässt, geht es in diesem Kurs \emph{nicht} darum, euch und uns zu demokratischeren Menschen zu erziehen --- wenn überhaupt, dann ist die Frage eher umgekehrt, wie Schule demokratischer werden könnte.
% MH paralleler Satz um was es nicht in der Pädagogik geht?
Schließlich geht es in unserem Kurs auch \emph{nicht} um das, was man gemeinhin unter Politik versteht --- vor allem keine Tages- oder Schulpolitik --- sondern allgemeiner darum, was eigentlich kollektiv verbindlichen Entscheidungen Legitimität verleihen könnte.

Wir arbeiten beide schon seit ein paar Jahren an unseren jeweiligen Themen: Verena an inklusiver Pädagogik, also dem gemeinsamen Lernen von unterschiedliche Menschen und Max an deliberative Demokratie, also der Entscheidungsfindung durch den Austausch von Begründungen.
In unseren Gesprächen ist uns deutlich geworden, dass diese beiden verschiedenen Themen und disparaten Disziplinen einige gemeinsame Fragen aufwerfen.
Eine dieser Fragen ist:
Wie können unterschiedliche und unterschiedlich \emph{fähige} Menschen zusammen leben und entscheiden, und dabei persönlich selbständig und inhärent gleich(wertig) bleiben, oder werden?
Das ist eine ziemlich große, und ziemlich abstrakte Frage.
Solche Fragen lassen sich besser bearbeiten, wenn wir sie an einem (einigermaßen) konkreten Betrachtungsgegenstand untersuchen.
Diese Betrachtungsgegenstände sind für uns Schule und Demokratie.

Das klingt erstmal recht abgehoben, und das ist es auch.
Wir haben in unserem Studium die Erfahrung gemacht, das es manchmal die Mühe Wert ist, sich auf abstrakte Gedanken und schwierige Texte einzulassen, die scheinbar wenig mit konkreten, fassbaren Dingen zu tun haben.
Zum einen kann Wissenschaft als Geistessport Freude bereiten, besonders, wenn man, wie wir in ``Schule der Demokratie'' einen Überblick über verschiedene Zeiten und Disziplinen akademischen Wirkens bekommt.
Zum anderen sind Theorie und Praxis \emph{durchaus} gleich --- jedenfalls theoretisch, wie der amerikanische Baseball-Trainer Yogi Berra einmal anmerkte.
Aphorismen beiseite gilt dies besonders für ``Schule der Demokratie'', in dem wir uns schließlich mit zwei einigermaßen radikal-progressiven Reformvorschlägen befassen: inklusive Pädagogik und deliberative Politik.
Wer nun --- wie vielleicht wir und ihr --- ganz praktisch etwas verändern will, der muss wohl erstmal wissen, \emph{was} denn verändert werden soll, und \emph{warum}.
Das sorgfältiges Nachdenken über Abstraktionen, das Hadern mit Begriffen und das Verständnis von anderer Leute Ideen ermöglicht es manchmal erst, Forderungen zu klären, und, vor allem, \emph{neue} Möglichkeiten zu sehen.

Bis in ein paar Wochen die intensivere Kursvorbereitung beginnt, möchten wir Dir den 2008 bei Carl Hanser erschienen Briefwechsel von Hartmut von Hentig empfehlen, in dem er seinem Neffen Tobias erklärt ``Warum muss ich zur Schule gehen?'' --- oder eben nicht.
% MH noch ein paar Sätze zu Hentig oder ok so?
Das Buch scheint momentan vergriffen, ist aber im Internet (etwa: \url{http://www.booklooker.de}) günstig gebraucht zu kaufen.
Wir stellen es in den nächsten Tagen auch als PDF bereit, und schicken Dir den Link dann per E-Mail.

Wenn Du Fragen hast, schreib' uns einfach.
Am besten sind wir per E-Mail zu erreichen, wenn es dringendes zu klären gibt, kannst Du uns aber auch gerne anrufen. Die Kontaktdaten stehen am Anfang dieses Briefes.
Wir freuen uns, von Dir zu hören.

Wir beide planen nun schon seit ein paar Jahren an diesem gemeinsamen Kurs und sind gespannt auf Dich, auf Deine Ideen und Beiträge, auf die Gruppe insgesamt, auf unsere gemeinsame Arbeit und Ergebnisse und natürlich auch auf die schönen Momente außerhalb des Seminarraums mit den anderen Teilnehmenden!

In Vorfreude auf die gemeinsame Zeit im Sommer,

\closing{
	Herzliche Grüße

	Deine Kursleiter
}

\end{letter}

\end{document}
