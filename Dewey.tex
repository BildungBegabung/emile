%!TEX root=./Emile.tex
\section[Epilog]{Dewey's Plädoyer für den Fortschritt}

Wie der Romantiker und Republikaner \citeauthor{rousseau-1762}, denkt auch der Philosoph, Sozialwissenschaftler und Pädagoge \citeauthor{Dewey2010} in seinem Werk Demokratie und Schule zusammen.
Vom Skeptiker der Moderne \citeauthor{Rousseau-1762-b} unterscheidet ihn aber seine pragmatische Fortschritts-Ethik.
Für \citeauthor{Dewey2010} müssen fortschrittliche Ideen des guten Handelns zwar immer \emph{in} und \emph{aus} der praktischen Erfahrung der Gegenwart (etwa: menschenwürdiges Einkommen) formuliert sein, sie haben aber immer über diese gegenwärtige Praxis hinaus zu deuten, und dürfen keinesfalls \emph{realistisch} einem Status Quo idealisieren.
Es wäre nach \citeauthor{Dewey2010} also fatal, würde man seine ethischen Vorstellungen der gegenwärtigen Realität anpassen.


\paragraph{Theorie des guten Handelns}

\citeauthor{Dewey1932}'s pragmatische Ethik fällt zwischen zwei andere Theorien des guten Handelns:

\begin{enumerate}
	\item Eine \emph{deontologische Ethik} verschreibt transzendentale (außerweltliche) Pflichten oder Tugenden ausgeht, deren Erfüllung gut ist, auch \emph{unabhängig} von den (materiellen) Konsequenzen dieses Handelns.
	\item Eine \emph{konsequentialistische Ethik} geht andererseits davon aus, dass Handeln gut ist, wenn die (wie auch immer operationalisierten) Ergebnisse gut sind.
\end{enumerate}

Der Pragmatismus bildet hierbei die Synthese zwischen diesen beiden sich widersprechenden Theorien.
Dieser sagt nämlich aus, dass wir zwar unsere Ethik immer auch in praxisrelevanten Kategorien (etwa: materiellen Konsequenzen) definiert sein muss, sich aber trotzdem nie in der Beschreibung dieser Realität begrenzen darf.
Tatsächlich muss eine pragmatische Ethik stets angepasst werden, wenn sich der Horizont der Praxis erweitert.


\paragraph{Deweys gesellschaftliches Ideal}

\citeauthor{Dewey2010} sieht den Fortschritt vom Naturzustand zur Demokratie als einen Suchprozess nach dem Ideal.
Fortschritt kann weiterhin auch durch Angst bewirkt werden.
Zum einen sieht \citeauthor{Dewey2010} Angst als Druckmittel der staatlichen Gewalt, zum anderen aber auch als Motivation, Aufgaben besser auszuführen.
Daraus ist abzuleiten, dass Angst nicht grundsätzlich als schlecht angesehen wird.
Sie wird allerdings häufig auf falsche Art und Weise institutionalisiert und verliert somit ihren möglicherweise positiven Effekt.
Im Sinne der Motivation kann mit Angst allerdings ein Problem auftreten, wenn z.B.\ die Arbeiter nicht wissen, warum sie etwas machen, sondern nur in Erwartung von Belohnung bzw.\ Bestrafung agieren.
Findet eine Mitarbeiterin keine Form von Selbstverwirklichung in ihrem Beruf, sondern arbeitet nur nach dem Prinzi von Belohnung und Strafe, so existiert sie nach \citeauthor{Dewey2010} in legaler Sklaverei --- oder \emph{entfremdete Arbeit}.
Wie sich an der folgenden Kritik an Platos Klassengesellschaft erkennen lässt, sind dabei laut \citeauthor{Dewey2010} der Individualismus, die Chancengleichheit und der Meinungsaustausch in einer Demokratie wichtig, damit es in der Gesellschaft die ständige Möglichkeit zur Neuausrichtung und Verbesserung gibt:

\begin{quote}
	``He never got any conception of the indefinite plurality of activities which may characterize an individual [...] and consequently limited his view to a limited number of \emph{classes}. [...]
	Hence education would soon reach a static limit in each class, for only diversity makes change and progress.''\\*
	\parencite[95f.]{Dewey-1916}
\end{quote}

Des Weiteren wird idealerweise die politische Mitwirkung jeder Einzelnen gewährleistet.
Alle müssen gehört werden, denn sie könnten etwas sehen.
Die Gesellschaft darf also nicht in Klassen geteilt werden, zwischen welchen kein Austausch stattfindet und die somit verhindern, dass freie Interaktion zwischen ihren Mitgliedern stattfinden kann.

Hier findet sich ein deutlicher Unterschied zu Rousseau, der nicht den Fortschritt als Ideal betrachtet, sondern den Naturzustand des Menschen.


\paragraph{Deweys Erziehungsideal}

\citeauthor{Dewey2010} geht davon aus, dass das Subjekt durch die Erziehung und die Gesellschaft bestimmt wird, denn ``jede Erziehung in einer Gruppe und durch eine Gruppe wirkt sozialisierend'' \parencite[115]{Dewey2010}.
Des Weiteren ist ``die Demokratie [...] mehr als eine Regierungsform, sie ist in erster Linie eine Form des Zusammenlebens, der gemeinsamen und miteinander geteilten Erfahrung.'' \parencite[121]{Dewey2010}.
Das Verständnis von Demokratie als Lebensform ist für \citeauthor{Dewey2010} grundlegend für sein Erziehungsverständnis.
Die Demokratie muss auf einer dynamischen Gesellschaft basieren und als Aufgabe, Projekt oder auch Ziel von allen gesehen werden.
Davon ausgehend, lässt sich aus Deweys pädagogischen Theorien schlussfolgern, dass er gar keinen Konflikt zwischen persönlicher Autonomie und inhärenter Gleichwertigkeit gibt.
Beides ist für eine demokratische Erziehung nach Dewey notwendig, und ergänzt sich \parencite[121]{Dewey2010}.
%VK FIXME: Zusammenhang herstellen
	%MH oh ja, das wäre schön. ``Maybe [next] time...'' (F. Sinatra)