\section{Einführung}

In unserem Kurs wird das Unmögliche versucht:
Wir bauen Brücken, wo sie noch nie gebaut worden sind.
Am Anfang unserer Kursarbeit hatten wir 16 Autoren sowohl der Sozialwissenschaften als auch der Pädagogik und ein utopisches Ziel: Alle miteinander zu verbinden.

Ein Grundproblem, das sich sowohl in den Sozialwissenschaften als auch der Pädagogik wiederfindet ist das des menschlichen Zusammenlebens.
In unsererm Kurs wurde diese Problematik, die sich in einer Demokratie ebenso wie in der Schule stellt, anhand eines Hauses veranschaulicht.

%![Haus](img\Kooperationshaus.JPG "Kooperationshaus")

Nehmen wir an, dass der Mensch weder Herdentier noch Einzelgänger ist.
Da aber (fast) jeder Mensch in eine Gesellschaft hineingeboren wird, ergibt sich zwangsläufig folgender Konflikt:
Inwiefern kann ein Mensch autonom über sich und sein Leben entscheiden, ohne dabei die Entscheidungsfreiheit anderer und somit die inhärente Gleichwertigkeit aller Menschen einzuschränken?

Die Aufgabe, die Balance zwischen persönlicher Autonomie und inhärenter Gleichheit zu finden, wird von unseren Autoren unterschiedlich gelöst.
Entsprechend unterschiedlich sind auch die Menschenbilder, die diesen Texten zugrunde liegen.
Sie bilden das Fundament unseres Hauses.

Den Herausforderungen des Zusammenlebens und -arbeitens nimmt sich dieser Kurs ganz konkret an:
Auf der Quellkontroll- und Versionsverwaltungsplattform "github.com" versuchen wir uns an einem selbstorganisiertem Modus der Zusammenarbeit - irgendwo zwischen Schule und Demokratie, zwischen Markt und Staat.
Dabei stellt sich uns immer wieder eine Frage:

Wie ist es also möglich, dass unterschiedliche und verschieden befähigte Menschen inhärent gleich und persönlich autonom zusammenleben?

Unsere Einstellung zur Kursarbeit lässt sich durch ein Zitat von Martina Gedeck treffend beschreiben:

% >"Im Leben geht es darum, die richtigen Fragen zu stellen und nicht darum, dauernd Antworten zu geben."
% <!-- FIXME: VK Beleg? -->

Die richtige Frage ist gestellt, der Grundstein für unsere Brücken ist gelegt.
