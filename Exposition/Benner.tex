%!TEX root=../Emile.tex

\subsection[Benner]{Benner: Einführung in die allgemeine Pädagogik}

\epigraph{
	``Hilf mir, es selbst zu tun.''\\*
	\emph{Maria Montessori}
}

\citeauthor{benner-2012} zeigt die \emph{Grundsätze pädagogischen Denkens und Handelns} auf.
Er richtet sich dabei an die pädagogisch interessierte Leserschaft.
Dazu definiert er zwei grundsätzliche große Extrema der Verhaltensbestimmung: Den \emph{Determinismus und die transzendentale Freiheit}.

\begin{enumerate}
	\item Nach dem \emph{Determinismus} ist das menschliche Verhalten grundsätzlich durch seine Umwelt \emph{oder} durch seine Anlagen vorher bestimmt.
	\item Im Sinne der \emph{transzendentalen Freiheit} kann der Mensch zu jedem Zeitpunkt vollkommen willkürlich über seine Handlungen und sein Schicksal entscheiden.
\end{enumerate}

\citeauthor{benner-2012} stimmt weder der einen, noch der anderen Position zu \citeyear{benner-2012}(vgl. S.71).

Während \citeauthor{siebert-2003} Strukturdeterminiertheit und Rekursivität als zentrale lerntheoretische Begriffe im Sinne der konstruktivistischen Pädagogik anführt, geht \citeauthor{benner-2012} erst einmal grundlegend davon aus, dass jeder Mensch \emph{bildsam} ist.
Bei der Bildsamkeit handelt es sich um eine deontologische Annahme oder Ethik, also einen ``letzten Grund''.
\citeauthor{benner-2012} ist der Meinung, ``Die Bildsamkeit eines Menschen anerkennen  heißt [...] nicht einen Mittelweg zwischen diesen Extremen [...] zu suchen'' \parencite[72]{benner-2012}.
%MH TODO: sondern...?

\citeauthor{benner-2012} gibt keine Gründe für die Bildsamkeit, sondern behauptet, sie liege grundsätzlich in der  Natur der Kommunikation zwischen Erzieherin und Zögling.
Schließlich meint Bildsamkeit, dass jeder Mensch dazu fähig sei, sich selbst zu bilden.
Bildsamkeit ist mehr als Potenzial zu verstehen und muss daher zur Entfaltung gebracht werden ––– würde es dazu immer der Pädagogik bedürfen, bliebe der Mensch schließlich in Abhängigkeit.
Der Zögling muss also von der Erzieherin zur Selbsttätigkeit aufgefordert werden.
Dies geschieht nach \citeauthor{benner-2012} \emph{intersubjektiv} in einem Erziehungsprozess.
% MH TODO das ist etwas unelegant; Aufforderung von Nase A nach Nase B muss IMMER intersubjektiv sein ... was genau ist hier also nennenswerterweise intersubjektiv?
Anders als nach Rousseau bedarf ein Mensch, laut \citeauthor{benner-2012}, pädagogischer Hilfe, falls er sein Potential nicht ausnutzt.
Somit ist ein Mensch nicht immer selbsttätig und umfassend bildsam, sondern braucht auch im Erwachsenenalter zum Teil Unterstützung \parencite[vgl.][91]{benner-2012}.
Hieraus ergibt sich für \citeauthor{benner-2012} die ``Imperfektheit des Menschen'' \citeyear[vgl.][78]{benner-2012}.

Die Bildsamkeit, als Prinzip menschlichen Lebens und Interagierens, ist entscheidend somit für die Frage nach persönlicher Autonomie und inhärenter Gleichwertigkeit, da Bildsamkeit dem Menschen grundsätzliche ein großes Maß an Autonomie zugesteht.
Obwohl nach \citeauthor{benner-2012} die Aufgabe der Erziehung auch weit über das Kindesalter hinaus geht, gesteht er jedem Individuum die Fähigkeit der Entscheidung zu, was er von seinem Leben will.
\citeauthor{benner-2012} sieht jedoch ein, dass die Freiheit eines Menschen nicht unendlich ist, schließlich sind Dinge wie Beeinflussung durch Anlagen und die Umwelt durchaus auch entscheidend für das Verhalten eines Menschen.
Daraus zeigt sich, dass das Prinzip der \emph{Bildsamkeit das Kooperationsproblem im Sinne der Pädagogik hinreichend löst}.

Bildsames Lernen basiert laut \citeauthor{benner-2012} auf vier Grundlagen, die nur dem Menschen vorbehalten sind.
Sie lauten:

\begin{enumerate}
	\item Leiblichkeit,
	\item Freiheit,
	\item Geschichtlichkeit und
	\item Sprachlichkeit.
\end{enumerate}

\citeauthor{benner-2012} ist der Auffassung:

\begin{quote}
	``Die pädagogische Praxis hat keine andere Möglichkeit [...] als den der Erziehung Bedürftigen als jemanden zu begegnen, der bildsam im Sinne der rezeptiven und spontanen Leiblichkeit, Freiheit, Geschichtlichkeit und Sprachlichkeit menschlicher Praxis ist.''\\*
	\textcite[vgl.][76]{benner-2012}.
\end{quote}

Leiblichkeit beschreibt das sensomotorische menschliche Handeln, denn besonders durch physische Interaktion ist es uns möglich, Zusammenhänge zu verstehen.
Mit Freiheit meint \citeauthor{benner-2012} im Grunde die menschliche Bildsamkeit als solches.
Und mit der Geschichtlichkeit die Abhängigkeit des Menschen von seinen Erfahrungen, die sein Verhalten in der Gegenwart beeinflussen.
Darüber hinaus sieht \citeauthor{benner-2012} in der Sprachlichkeit das wichtige Mittel zur Interaktion.

Als Beispiel für einen Menschen, bei dem dieser Prozess gescheitert ist, bezieht sich \citeauthor{benner-2012} auf ein Beispiel von J. Jegge.
%MH TODO für VK: evtl vollständige Zitation einarbeiten?
Es handelt sich dabei um eine Geschichte über einen Jungen namens Heini \parencite[vgl.][74]{benner-2012}.
Dieser hat das Potenzial seiner Bildsamkeit nicht ausgenutzt.
Sichtbar wird dies an seiner Unfähigkeit, sich etwas frei und nur auf sich selbst bedacht zu wünschen.
Wenn man ihm eine Frage stellt, was er von zwei Dingen lieber hätte, so antwortet er stets mit Rückfragen, die ihm dabei helfen sollen zu verstehen, was der Erzieher will, was er wollen soll.
Er tut dies, weil er in seiner Vergangenheit immer Strafe rechnen musste, wenn er nicht das getan hat, was andere von ihm verlangt haben.
Es ist nun notwendig, so Benner, ihm wieder Bildsamkeit und das eigene Wünschen zu zeigen und ihn zur Selbsttätigkeit, seine \emph{eigenen} Wünsche auszudrücken, zu befähigen \parencite[vgl.][20]{benner-2012}.

Dieser Ansatz definiert den Menschen als grundsätzlich gleichwertig, in seinem Potenzial bildsam zu sein.
Er wird also als etwas gesehen, was er nicht ist.
\citeauthor{benner-2012} leitet daraus ab, dass Metakommunikation bzw. die aktive Reflektion des Zöglings über seine Gedankenprozesse entscheidend ist um das Potenzial der Bildsamkeit zu nutzen.
