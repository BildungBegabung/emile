%!TEX root=../Emile.tex

\subsection[Benner]{Benners Einführung in die allgemeine Pädagogik}

\epigraph{
	``Hilf mir, es selbst zu tun.''\\*
	\emph{Maria Montessori}
}

\citeauthor{benner-2012} definiert zwei grundsätzliche Extrema der Verhaltensbestimmung, von denen er seine \emph{Allgemeine Pädagogik} abgrenzt: den Determinismus und die transzendentale Freiheit.

\begin{enumerate}
	\item Nach dem \emph{Determinismus} ist das menschliche Verhalten grundsätzlich durch seine Umwelt \emph{oder} durch seine Anlagen vorher bestimmt.
	\item Im Sinne der \emph{transzendentalen Freiheit} kann der Mensch zu jedem Zeitpunkt vollkommen willkürlich über seine Handlungen und sein Schicksal entscheiden.
\end{enumerate}

\citeauthor{benner-2012} stimmt weder der einen noch der anderen Position zu \parencite[vgl.][71]{benner-2012}.

Während \citeauthor{siebert-2003} Strukturdeterminiertheit und Rekursivität als zentrale lerntheoretische Begriffe im Sinne der konstruktivistischen Pädagogik anführt, geht \citeauthor{benner-2012} grundlegend davon aus, dass jeder Mensch \emph{bildsam und selbsttätig} ist.
\citeauthor{benner-2012} ist der Meinung, ``Die Bildsamkeit eines Menschen anerkennen heißt [...], nicht einen Mittelweg zwischen diesen Extremen [...] zu suchen'' \parencite[72]{benner-2012}, sondern die Zu-Erziehende aktiv, d.h. selbsttätig in den Erziehungsprozess miteinzubeziehen.
Als deontologische Annahme, oder ``letzter Grund'' liege die Bildsamkeit grundsätzlich in der Natur des Zöglings.
Bildsamkeit und Selbsttätigkeit, als zwei Seiten einer Medaille, können als Potenziale des Menschen angesehen werden, die im Laufe seines Lebens zur Entfaltung gebracht werden.
Laut \citeauthor{benner-2012} bedarf er dafür pädagogischer Hilfe.
Trotz der ``Imperfektheit des Menschen'' \parencite[vgl.][78]{benner-2012} und seinem Bedürfnis nach pädagogischer Unterstützung zum Teil auch im Erwachsenenalter (etwa bei Lebenskrisen), ist Erziehung (im Gegensatz zur Politik und Ökonomie) auf ihr eigenes Ende hin ausgerichtet \parencite[vgl.][90f.]{benner-2012}.
Würde der Mensch ein Leben lang anhaltend Erziehung benötigen, bliebe er ewig in Abhängigkeit, was im Widerspruch zur Selbsttätigkeit stehen würde.

Bildsamkeit und Selbsttätigkeit, als zwei Prinzipien pädagogischen Denkens und Handelns, sind entscheidend für die Grundfrage unseres Kurses nach persönlicher Autonomie und inhärenter Gleichwertigkeit.
In der Bildsamkeit erscheint, pädagogisch betrachtet, inhärente Gleichwertigkeit.
Selbsttätigkeit bestätigt den Anspruch möglichst autonomer Lebensführung jedes Individuums.
\citeauthor{benner-2012} sieht jedoch ein, dass die Autonomie eines Menschen nicht unendlich ist, schließlich sind Dinge wie Beeinflussung durch Anlagen und die Umwelt auch entscheidend für das Verhalten eines Menschen.

Bildsames Lernen basiert laut \citeauthor{benner-2012} auf vier Grundlagen, die nur dem Menschen vorbehalten sind.
Sie lauten:

\begin{enumerate}
	\item Leiblichkeit,
	\item Freiheit,
	\item Geschichtlichkeit und
	\item Sprachlichkeit.
\end{enumerate}

\citeauthor{benner-2012} ist der Auffassung:

\begin{quote}
	``Die pädagogische Praxis hat keine andere Möglichkeit [...] als den der Erziehung Bedürftigen als jemanden zu begegnen, der bildsam im Sinne der rezeptiven und spontanen Leiblichkeit, Freiheit, Geschichtlichkeit und Sprachlichkeit menschlicher Praxis ist.''\\*
	\parencite[vgl.][76]{benner-2012}.
\end{quote}

Leiblichkeit beschreibt das sensomotorische menschliche Handeln, denn besonders durch physische Interaktion ist es uns möglich, Zusammenhänge zu verstehen.
Mit Freiheit meint \citeauthor{benner-2012} im Grunde die menschliche Bildsamkeit als solches.
Und mit der Geschichtlichkeit die Abhängigkeit des Menschen von seinen Erfahrungen, die sein Verhalten in der Gegenwart beeinflussen.
Darüber hinaus sieht \citeauthor{benner-2012} in der Sprachlichkeit das wichtige Mittel zur Interaktion.

Als Beispiel für einen Menschen, bei dem dieser Prozess gescheitert ist, bezieht sich \citeauthor{benner-2012} auf ein Beispiel von Jürg Jegge.
%MH TODO für VK: evtl vollständige Zitation einarbeiten?
Es handelt sich dabei um eine Geschichte über einen Jungen namens Heini \parencite[vgl.][nach Jegge 1976][74]{benner-2012}.
Dieser hat das Potenzial seiner Bildsamkeit nicht ausgeschöpft.
Sichtbar wird dies an seiner Unfähigkeit, sich etwas frei und nur auf sich selbst bedacht zu wünschen.
Wenn man Heini eine Frage stellt, was er von zwei Dingen lieber hätte, so antwortet er stets mit Rückfragen, die ihm dabei helfen sollen herauszufinden, was der Erzieher will, was er \emph{wollen soll}.
Er tut dies, weil er in seiner Vergangenheit immer mit Strafen rechnen musste, wenn er nicht das getan hat, was andere von ihm verlangt haben.
Es ist nun notwendig, so \citeauthor{benner-2012}, Heini wieder Bildsamkeit und das eigene Wünschen zu zeigen und ihn zur Selbsttätigkeit, seine \emph{eigenen} Wünsche auszudrücken, zu befähigen \parencite[vgl.][20]{benner-2012}.
Damit muss er als etwas angesehen werden, was er nicht ist.

Am Beispiel von Heini zeigt sich das pädagogische Grundproblem, dass jede Zu-Erziehende als \emph{bildsam und selbsttätig} angesehen werden muss, ohne dass sich dies tatsächlich schon im Verhalten ausdrückt \parencite[vgl.][90]{benner-2012}.
Interventionen durch Erziehung stehen damit immer in der Gefahr, dem Ziel (Bildsamkeit und Selbsttätigkeit) zuwiderlaufende Mittel einzusetzen (Zwang) und damit einen Kreislauf der Abhängig in Gang zu bringen.
\citeauthor{benner-2012} sieht eine Lösung für dieses Paradox in der Metakommunikation bzw. der gemeinsamen Reflexion mit dem Zögling über seine Gedankenprozesse und seinen Lernprozess.