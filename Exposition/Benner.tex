%!TEX root=./Emile.tex
\subsection{Benner: Einführung in die allgemeine Pädagogik}

\epigraph{
		``Hilf mir, es selbst zu tun''}
	{
		\emph{Maria Montessori
	}

\citeauthor{benner-2012} setzt sich damit auseinander, die **Grundsätze pädagogischen Handelns** aufzuzeigen, die für die ganze Menschheit gültig sind.
Er richtet sich dabei an die pädagogisch~ interessierte Leserschaft.
Dazu definiert er zwei grundsätzliche große Extrema der Verhaltensbestimmung: Den \emph{Determinismus und die transzendentale Freiheit}.
Nach dem Determinismus ist das menschliche Verhalten grundsätzlich durch seine Umwelt oder seine Anlagen vorher bestimmt.
Im Sinne der transzendentalen Freiheit kann der Mensch zu jedem Zeitpunkt vollkommen willkürlich über seine Handlungen und sein Schicksal entscheiden.
\citeauthor{benner-2012} stimmt weder dem Einen, noch dem Anderen zu (vgl. S.71).
Während \citeauthor{siebert-2003} als zentralen theoretischen Ansatz den Konstruktivismus anführt, ist \citeauthor{benner-2012}der Auffassung, dass jeder Mensch \emph{bildsam} ist.
Bei dieser Bildsamkeit handelt es sich um eine deontologische Annahme.
\citeauthor{benner-2012} ist der Meinung, ``Die Bildsamkeit eines Menschen anerkennen  heißt [...] nicht einen Mittelweg zwischen diesen Extremen [...] zu suchen'' \parencite[72]{benner-2012}.

\citeauthor{benner-2012} gibt keine Gründe für die Bildsamkeit, sondern behauptet, sie liege grundsätzlich in der  Natur der Kommunikation zwischen Edukatorin und Edukandin.
Schließlich meint Bildsamkeit, dass jeder Mensch dazu fähig ist sich selbst zu bilden.
Allerdings ist diese Bildsamkeit mehr ein Potenzial und muss daher von der Edukatorin in der Edukandin zur Entfaltung gebracht werden.
Der Mensch muss zur Selbsttätigkeit aufgefordert werden.
Dies geschieht nach \citeauthor{benner-2012} intersubjektiv in einem Erziehungsprozess.
Da es sich hierbei um einen Vorgang handelt, kann ein Mensch nur als vollendet erzogen angesehen werden, wenn er selbstständig und bildsam handelt.
Anders als nach Rousseau bedarf ein Mensch, laut \citeauthor{benner-2012}, pädagogischer Hilfe falls er sein Potential nicht ausnutzt.
Somit ist ein Mensch nicht unendlich erwachsen, sondern nur so lange, wie er sich auch so verhält \parencite[vgl.][91]{benner-2012}.
\citeauthor{benner-2012} nennt es die ``Imperfektheit des Menschen'' \parencite[vgl.][78]{benner-2012}.

Die Bildsamkeit, als Prinzip menschlichen Lebens und Interagierens, ist entscheidend somit für die Frage nach persönlicher Autonomie und inhärenter Gleichwertigkeit, da Bildsamkeit dem Menschen grundsätzliche ein großes Maß an Autonomie zugesteht.
Obwohl nach \citeauthor{benner-2012} die Aufgabe der Erziehung auch weit über das Kindeslter hinaus geht, gesteht er jedem Individuum die Fähigkeit der Entscheidung zu, was er von seinem Leben will.
\citeauthor{benner-2012} sieht jedoch ein, dass die Freiheit eines Menschen nicht unendlich ist, schließlich sind Dinge wie Beeinflussung durch Anlagen und die Umwelt durchaus auch entscheidend für das Verhalten eines Menschen.
Daraus zeigt sich, dass das Prinzip der **Bildsamkeit das Kooperationsproblem im Sinne der Pädagogik hinreichend löst**.

Bildsames Lernen basiert laut \citeauthor{benner-2012} auf vier Grundlagen, die nur dem Menschen  vorbehalten sind.
Sie lauten: Leiblichkeit, Freiheit, Geschichtlichkeit und Sprachlichkeit.
\citeauthor{benner-2012} ist der Auffassung:
``Die pädagogische Praxis hat keine andere Möglichkeit [...] als den der Erziehung Bedürftigen als jemandenden zu begegnen, die Bildsam im Sinne der rezeptiven und spontanen Leiblichkeit, Freiheit, Geschichtlichkeit und Sprachlichkeit menschlicher Praxis sind'' \parencite[vgl.][76]{benner-2012}.

Leiblichkeit beschreibt das sensomotorische menschliche Handeln, denn besonders durch physische Interaktion ist es uns möglich Zusammenhänge zu verstehen.
Mit Freiheit meint \citeauthor{benner-2012} im Grunde die menschliche Bildsamkeit als solches und mit der Geschichtlichkeit die Abhängigkeit des Menschen von seinen Erfahrungen, die sein verhalten in der Gegenwart beeinflussen.
Darüber hinaus sieht \citeauthor{benner-2012} in der Sprachlichkeit das wichtige Mittel zur Interaktion.

Als Beispiel für einen Menschen, bei dem dieser Prozess gescheitert ist, bezieht sich \citeauthor{benner-2012} auf ein Beispiel von Jegge, einen Jungen namens Heini \parencite[vgl.][74]{benner-2012}.
Dieser hat das Potenzial seiner Bildsamkeit überhaupt nicht ausgenutzt.
Sichtbar wird dies an seiner Unfähigkeit, sich etwas frei und nur auf sich selbst bedacht zu wünschen.
Wenn man ihm eine Frage stellt, was er von zwei Dingen lieber hätte, so antworte er stets mit Rückfragen, die ihm dabei helfen sollen zu verstehen, was der Erzieher will, was er wollen soll.
Er tue dies, weil er in seiner Vergangenheit immer Strafe rechnen musste, wenn er nicht das getan hat, was andere von ihm verlangt haben.
Es ist nun notwendig, so Benner, ihm wieder Bildsamkeit und das eigene Wünschen zu zeigen \parencite[vgl.][20]{benner-2012}.

Dieser Ansatz definiert den Menschen grundsätzlich in seinem Potenzial, er wird also als etwas gesehen, was er nicht ist.
\citeauthor{benner-2012}leitet daraus ab, dass Metakommunikation bzw. die aktive Reflektion des Zöglings über seine Gedankenprozesse entscheidend ist um das Potenzial der Bildsamkeit zu nutzen.
