# Warum Demokratie?

> "Die Demokratie ist die schlechteste aller Staatsformen, ausgenommen aller Anderen"
> (Winston Churchill

Nach Tilly lässt sich festhalten, dass ein Staat von dem Punkt an existiert, ab dem sich ein Gewaltmonopol einer Guppe auf einem bestimmten Gebiet gebildet hat, welches Sicherheit sowohl gegen sich selbst, als auch gegen Andere bietet.
Aus dieser Annahme ergiebt sich Folgende Fragestellung:
Wie lässt sich das menschliche Zusammenleben in einem solchen Staat organisieren?
Diese Organisation, muss nach Dahl die drei "Dimensionen menschlichen Intresses" bestmöglich umsetzten:

  1. Entwicklung und Entfaltung menschlicher Fähigkeiten ( vgl. Dahl-1989 S. 88)
  2. Befriedigung der Interessen des Individuums ( ebd. S. 88)
  3. Maximal mögliche Freiheit (ebd. S. 88)

Dahl geht davon aus, dass Demokratie als Organisationsprinzip am besten geeignet ist um menschliches Zusammenleben zu koordinieren und die drei Dimensionen menschlicher Interessen zu erfüllen.
Er versucht darüber hinaus Demokratie als Staatsform zu legitimieren, da sie sowohl persönliche Autonomie, als auch intrinsiche Gleichwertigkeit ermöglicht.


## Intrinsische Gleichheit & Persönliche Autonomie

Dahls Rechtfertigung von **Demokratie** stützt sich auf die **Kombination aus intrinsischer Gleichwertigkeit und persönlicher Autonomie**.
Intrinsische Gleichwertigkeit bedeutet, dass die Interessen eines jeden Menschen grundsätzlich gleichwertig sind und daher bei jeder Entscheidung auch gleichermaßen in Betracht gezogen werden müssen.
Persönliche Autonomie beschreibt die Entscheidungsfreiheit, dass jeder die Entscheidungen treffen kann, die für sie am besten ist: "every adult is the best judge of his or her own interest" (Dahl-1989 S.100).
Zwischen diesen beiden Anschauungen ergibt sich ein Zusammenhang, da niemand so außerordentlich besser qualifiziert ist, dass sie alle Entscheidungen für eine bestimmten Gruppe von Personen treffen kann.

In Dahls Definition der intrinsischen Gleichheit sollen möglichst alle Menschen von einer Entscheidung profitieren, daher sind Entscheidungen, die zwar die Gemeinschaft fördern, aber einzelnen Individuen schaden, nicht vorgesehen.


## Kriterien für eine funktionierende Demokratie

Um die Demokratie zu sichern, setzt Dahl **vier Kriterien** voraus:

**Effektive Partizipation**, **gleiches Wahlrecht**, **Kontrolle der Tagesordnung** und **aufgeklärtes Verständnis**.
Das vierte Kriterium, für uns das wichtigste, bezieht sich auf die Notwendigkeit von Bildung für Demokratie:
"But to know what it wants, the people must be enlightened, at least at some degree" (ebd., S.100).
Allerdings darf ein Mitglied der Gemeinschaft nicht zur Bildung gezwungen werden, obwohl ein gewisses Interesse über politische Prozesse aufgeklärt zu beleiben vorausgesetzt wird.
Eine Verpflichtung würde zu einer Einschränkung der persönlichen Autonomie führen.
Trotzdem unterstütz Dahl die Bildung des Volkes und des demokratisches Verhalten:

>"It is foolish beacuse democracy has usually been concieved as a system in which "rule by the people" makes it more likely that people will get what it wants, or what it believes it is best, that alternative systems like guardianship in which an elite determines what is best"·
> (ebd., S.111, "enlightened understanding").


## Bezug zur Kursfrage

Dahl sieht in Demokratie einen Mittelweg zwischen absoluter persönlicher Autonomie und inherenter Gleichheit.
Er ist als liberaler Demokrat zwar der Auffassung, dass jeder Mensch sich an der Demokratie beteiligen muss, man aber niemanden dazu zwingen darf.
Dadurch räumt Dahl der persönlichen Autonomie einen sehr hohen Stellenwert ein.
Dahl würde sogar eher eine von allen gewollte Diktatur vorziehen, als eine erzwungene Demokratie.

>"Die Demokratie setzt die Vernunft im Volke voraus, die sie erst hervorbringen soll"
> (Karl Jasper)
