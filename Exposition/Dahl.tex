%!TEX root=./Emile.tex
\subsection{Dahl: Warum Demokratie?}

\epigraph{
		``Die Demokratie ist die schlechteste aller Staatsformen, ausgenommen aller Anderen''
		}
	{
		\emph{Winston Churchill
	}

Nach \citeauthor{Tilly-1985-aa} lässt sich festhalten, dass ein Staat von dem Punkt an existiert, ab dem sich ein Gewaltmonopol einer Gruppe auf einem bestimmten Gebiet gebildet hat, welches Sicherheit sowohl gegen sich selbst, als auch gegen Andere bietet.
Aus dieser Annahme ergibt sich folgende Fragestellung:
Wie lässt sich das menschliche Zusammenleben in einem solchen Staat organisieren?
Diese Organisation, muss nach \citeauthor{Dahl-1989-aa} die ``drei Dimensionen menschlichen Interesses'' bestmöglich umsetzten:

	\begin{enumerate}
		\item Die Entwicklung und Entfaltung menschlicher Fähigkeiten

		\item Befriedigung der Interessen des Individuums

		\item Maximal mögliche Freiheit \citep[88]{Dahl-1989-aa}
	\end{enumerate}

\citeauthor{Dahl-1989-aa} geht davon aus, dass Demokratie als Organisationsprinzip am besten geeignet ist um menschliches Zusammenleben zu koordinieren und die drei Dimensionen menschlicher Interessen zu erfüllen.
Er versucht darüber hinaus Demokratie als Staatsform zu legitimieren, da sie sowohl persönliche Autonomie, als auch inhärente Gleichwertigkeit ermöglicht.


\paragraph*{Inhärente Gleichwertigkeit & Persönliche Autonomie}

Dahls Rechtfertigung von Demokratie stützt sich auf die \emph{Kombination aus intrinsischer Gleichwertigkeit und persönlicher Autonomie}.
Intrinsische Gleichwertigkeit bedeutet, dass die Interessen eines jeden Menschen grundsätzlich gleichwertig sind und daher bei jeder Entscheidung auch gleichermaßen in Betracht gezogen werden müssen \citep[vgl.][100]{Dahl-1989-aa}.
Persönliche Autonomie beschreibt die Entscheidungsfreiheit, dass jeder die Entscheidungen treffen kann, die für sie am besten ist: ``every adult is the best judge of his or her own interest'' \citep[100]{Dahl-1989-aa}.
Zwischen diesen beiden Anschauungen ergibt sich ein Zusammenhang, da niemand so außerordentlich besser qualifiziert ist, dass sie alle Entscheidungen für eine bestimmten Gruppe von Personen treffen kann.

In Dahls Definition der intrinsischen Gleichheit sollen möglichst alle Menschen von einer Entscheidung profitieren, daher sind Entscheidungen, die zwar die Gemeinschaft fördern, aber einzelnen Individuen schaden, nicht vorgesehen.


\paragraph*{Kriterien für eine funktionierende Demokratie}

Um die Demokratie zu sichern, setzt \citeauthor{Dahl-1989-aa} vier Kriterien voraus:
	\begin{enumerate}
		\item Effektive Partizipation
		\item gleiches Wahlrecht
		\item Kontrolle der Tagesordnung
		\item aufgeklärtes Verständnis \citep[vgl.][100]{Dahl-1989-aa}
	\end{enumerate}

Aufgeklärtes Verständnis, für uns das wichtigste Kriterium, bezieht sich auf die Notwendigkeit von Bildung für Demokratie:
``But to know what it wants, the people must be enlightened, at least at some degree'' \citep[100]{Dahl-1989-aa}.
Allerdings darf ein Mitglied der Gemeinschaft nicht zur Bildung gezwungen werden, obwohl ein gewisses Interesse über politische Prozesse aufgeklärt zu bleiben vorausgesetzt wird.
Eine Verpflichtung würde zu einer Einschränkung der persönlichen Autonomie führen.
Trotzdem unterstützt \citeauthor{Dahl-1989-aa} die Bildung des Volkes und des demokratisches Verhalten:

	\begin{quote}
		``It is foolish beacuse democracy has usually been concieved as a system in which `rule by the people' makes it more likely that people will get what it wants, or what it believes it is best, that alternative systems like guardianship in which an elite determines what is best.'' \citep[111]{Dahl-1989-aa}
	\end{quote}


\paragraph*{Bezug zur Kursfrage}

\citeauthor{Dahl-1989-aa} sieht in Demokratie einen Mittelweg zwischen absoluter persönlicher Autonomie und inhärenter Gleichheit.
Er ist als liberaler Demokrat zwar der Auffassung, dass jeder Mensch sich an der Demokratie beteiligen muss, man aber niemanden dazu zwingen darf.
Dadurch räumt \citeauthor{Dahl-1989-aa} der persönlichen Autonomie einen sehr hohen Stellenwert ein.
\citeauthor{Dahl-1989-aa} würde sogar eher eine von allen gewollte Diktatur vorziehen, als eine erzwungene Demokratie.

	\begin{quote}
		``Die Demokratie setzt die Vernunft im Volke voraus, die sie erst hervorbringen soll'' \emph{Karl Jasper}
	\end{quote}