%!TEX root=./Emile.tex

\subsection[Sozialwahltheorie]{Sozialwahltheorie: Die unmögliche Perfektion der Demokratie}

\epigraph{
	``Those who vote count for nothing.
	Those who count the vote count for everything.''\\*
	\emph{Joseph Stalin}
}

Wenn eine Gruppe von Menschen eine Reihenfolge von Prioritäten durch ein Gruppenentscheid festlegen möchte, dann ist ein Wahlsystem notwendig.
Besonders bei \emph{kollektiv verbindlichen} Entscheidungen in einer Demokratie sollte darauf geachtet werden, dass alle Meinungen gleichwertig in das Endergebnis einfließen, wofür sich unterschiedliche Wahlsysteme anbieten.
Die Sozialwahltheorie evaluiert den Entscheidungsprozess von einem mathematischen Standpunkt aus, mit dem Ziel, ein perfektes Wahlsystem zu aufzustellen \parencite[vgl.][]{Kleinberg-2009-oz}.


\paragraph{Arrow's Unmöglichkeitstheorem}

Hierfür zieht Arrow drei Kriterien heran, die jeweils einen wünschenswerten Zusammenhang zwischen individuellen Präferenzen und Gruppenpräferenz beschreiben:

\begin{enumerate}
	\item \emph{Schwaches Pareto-Prinzip.} Wenn alle Wählerinnen Alternative $A$ über $B$ präferieren, dann ist auch die Gruppenpräferenz $A>B$.

	\item \emph{Unabhängigkeit von irrelevanten Alternativen.} (IIA) Die Gruppenpräferenzen zwischen zwei vorrangigen Alternativen bleiben unverändert, wenn sich individuelle Präferenzen lediglich hinsichtlich weiterer, nachrangiger Alternativen ändern.

	\item \emph{Nicht-Diktatur.} Keine Wählerin kann durch Änderung ihrer individuellen Präferenzordnung die Gruppenpräferenz diktieren.
\end{enumerate}

Wären alle drei Kriterien dieser Dreieckskonstellation erfüllt, läge ein perfektes Wahlsystem vor.
Aus dem \emph{Unmöglichkeitstheorem} geht jedoch hervor, dass, sobald mehr als zwei Alternativen zur Wahl stehen, kein Wahlsystem alle drei Kriterien garantieren kann \parencite[vgl.][748]{Kleinberg-2009-oz}.


\paragraph{Lösungsmöglichkeit}

Die Widersprüche des Unmöglichkeitstheorem von Arrow (etwa nicht-transitive Ergebnislisten) können abgemildert und Aggregationsdysfunktionen reduziert werden, wenn alle ordinal sortierten Präferenzlisten der Wählenden in einer für alle gleichen Ordnung der Optionen nur ein einziges Globales Maximum aufweisen (\emph{single-peakedness}).

Wenn man in der Politik Parteien zum Beispiel graduell nach politischer Orientierung (rechts/links) sortiert, würden Wähler, die eine rechte Partei an erster und ein Linke Partei an zweiter Stelle einsortieren, evtl. zu nicht transitiven Gruppenergebnissen beitragen.
Hierbei wird jedoch die Bestimmung der Kriterien des Wählers zum Problem.
Legt ein Wähler mehrere Kriterien an, entsteht sogar schon innerhalb seiner individuellen Präferenzliste dieses Problem.

Um solches Wahlverhalten zu vermeiden,

\begin{enumerate}
	\item muss eine Übereinstimmung aller Wählenden über die Ordnung der Alternativen bestehen.

	\item müssen alle Wählenden über diese Verteilung aufgeklärt sein.
\end{enumerate}

Die Frage ist, ob es überhaupt aus demokratischer Sicht vertretbar wäre, eine allgemein gültige Ordnung der Alternativen festzulegen, oder auch nur deren Hausbildung zu fördern.
Laut \citeauthor{Dahl-1989-aa} ist niemand offensichtlich besser qualifiziert als andere.
Folglich kann auch niemand allein befähigt sein, für die Allgemeinheit Alternativen anhand selbstgewählter Kriterien zu ordnen.
Somit verbirgt sich hinter den von der Sozialwahltheorie beschriebenen Dysfunktionen der liberalen Demokratie eine Lösung, die erneut auf die Widersprüche zwischen persönlicher Autonomie und inhärenter Gleichwertigkeit verweist.
Zwar könnten Bürgerinnen eher in die Lage versetzt werden, \emph{gleichwertig} und \emph{autonom} durch single-peaked Präferenzen zu wählen, eine entsprechende politische Bildung hin zu strukturierteren Präferenzen würde aber ihrerseits den gleichen, bekannten Widerspruch in der Schule der Demokratie hervorrufen.
