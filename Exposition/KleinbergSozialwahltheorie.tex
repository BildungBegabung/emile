%!TEX root=./Emile.tex

\subsection[Sozialwahltheorie]{Die unmögliche Perfektion der Demokratie}

\epigraph{
	``Those who vote count for nothing.
	Those who count the vote count for everything.''\\*
	\emph{Joseph Stalin}
}

Wenn eine Gruppe von Menschen eine Reihenfolge von Prioritäten durch ein Gruppenentscheid festlegen möchte, dann ist ein Wahlsystem notwendig.
Besonders bei \emph{kollektiv verbindlichen} Entscheidungen in einer Demokratie sollte darauf geachtet werden, dass alle Meinungen gleichwertig in das Endergebnis einfließen, wofür sich unterschiedliche Wahlsysteme anbieten.
Die Spieltheorie evaluiert den Entscheidungsprozess von einem mathematischen Standpunkt aus, mit dem Ziel, ein perfektes Wahlsystem zu aufzustellen \parencite[vgl.][]{Kleinberg-2009-oz}.


\paragraph{Arrow's Unmöglichkeitstheorem}

Hierfür zieht Arrow drei Kriterien heran:

\begin{enumerate}
	\item Bei Abstimmungen über Veränderungen muss die Situation von jedem mindestens unverändert oder besser sein, so dass eine \emph{Pareto Verbesserung} vorliegt.

	\item Die Gruppenpräferenzen zwischen zwei Alternativen bleiben unverändert, sobald eine dritte hinzugefügt wird, d.h. das \emph{IIA} bleibt unverletzt.
%VK: Def. IIA ergänzen
	\item Das Ergebnis muss eindeutig sein, und darf nicht, je nach Auslegung, unterschiedliche Ergebnisse produzieren, was eine Diktatur zur Folge hätte.
\end{enumerate}

In den USA (Mehrheitswahlrecht) kann zum Beispiel durch die Festsetzung der Wahlbezirke für das Repräsentantenhaus indirekt ein erheblicher Einfluss auf den Ausgang der Wahl genommen werden.

Wären alle drei Kriterien dieser Dreieckskonstellation erfüllt, läge ein perfektes Wahlsystem vor.
Aus dem \emph{Unmöglichkeitstheorem} geht jedoch hervor, dass, sobald mehr als zwei Alternativen zur Auswahl stehen, kein Wahlsystem alle drei Kriterien garantieren kann \parencite[vgl.][748]{Kleinberg-2009-oz}.


\paragraph{Lösungsmöglichkeit}

Die Widersprüche des Unmöglichkeitstheorem von Arrow (etwa: nicht-transitiver Ergebnislisten) können abgemildert, und Aggregationsdysfunktionen reduziert werden, wenn alle ordinal sortierten Präferenzlisten der Wählenden in einer für alle gleichen Ordnung der Optionen nur ein einziges Globales Maximum aufweisen (\emph{single-peakedness}).

Wenn man in der Politik Parteien zum Beispiel graduell nach politischer Orientierung (rechts/links) sortiert, würden Wähler, die eine rechte Partei an erster und ein Linke Partei an zweiter Stelle einsortieren, evtl. zu nicht transitiven Gruppenergebnissen beitragen.
Hierbei wird jedoch die Bestimmung der Kriterien des Wählers zum Problem.
Legt ein Wähler mehrere Kriterien an, entsteht sogar schon innerhalb seiner individuellen Präferenzliste dieses Problem.

Um solches Wahlverhalten zu vermeiden,

\begin{enumerate}
	\item muss eine Übereinstimmung aller Teilnehmer über die Ordnung der Alternativen bestehen.

	\item müssen alle Teilnehmer über diese Verteilung aufgeklärt sein.
\end{enumerate}

Die Frage ist, ob es überhaupt aus demokratischer Sicht vertretbar wäre, eine allgemein gültige Ordnung der Alternativen festzulegen, oder auch nur deren Hausbildung zu fördern.
Laut \citeauthor{Dahl-1989-aa} ist niemand offensichtlich besser qualifiziert als andere, folglich kann auch niemand allein befähigt sein, für die Allgemeinheit Alternativen anhand selbstgewählter Kriterien zu ordnen.
Somit verbirgt sich hinter den von der Sozialwahltheorie beschriebenen Dysfunktionen der liberalen Demokratie eine Lösung, die erneut auf die Widersprüche zwischen persönlicher Autonomie verweist.
Zwar könnten Bürgerinnen eher in die Lage versetzt werden, \emph{gleichwertig} und \emph{autonom} durch single-peaked Präferenzen zu wählen, eine entsprechende politische Bildung hin zu strukturierteren Präferenzen würde aber ihrerseits den gleichen, bekannten Widerspruch in der Schule der Demokratie hervorrufen.
