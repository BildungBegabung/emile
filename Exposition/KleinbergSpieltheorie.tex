%!TEX root=../Emile.tex

\subsection[Spieltheorie]{Spieltheorie: Was ist deine Strategie?}

\epigraph{
	``Gentlemen, Adam Smith needs revision.''\\*
	\emph{Fiktionaler John Nash im Kinofilm ``A Beautiful Mind'' (2001)}
}

Die \emph{Spieltheorie} modelliert menschliche Interaktion in Form von Spielen, dessen \emph{Spielausgänge} von den gewählten \emph{Spielstrategien} abhängen und in Form einer Auszahlungsmatrix analysiert und veranschaulicht werden können \parencite[vgl.][153-274]{Kleinberg-2009-oz}.

\begin{quote}
	``Game theory is concerned with situations in which decision-makers interact with one another, and in which the happiness of each participant with the outcome depends not just on his or her own decisions but on the decisions made by everyone.''\\*
	\parencite[vgl.][156]{Kleinberg-2009-oz}
\end{quote}

Bei einem \emph{Spiel} handelt es sich um eine Entscheidungsfindung von Spielern, in der alle Beteiligten wechselseitig von den Entscheidungen der anderen abhängig sind und stets nach Eigeninteresse handeln.
Ein gängiges Beispiel für ein Spiel der Spieltheorie liefert das \emph{Gefangenendilemma}.

\begin{dsafigure}
	\begin{center}
	\includegraphics[width=0.9\columnwidth]{img/gefangendilemma.jpg}
	\caption{Beispiel für ein Gefangenendilemma nach \textcite{Kleinberg-2009-oz}}
	\label{fig:gefangenendilemma}
	\end{center}
\end{dsafigure}

Die Abbildung oben stellt eine Auszahlungsmatrix dar.
Hier werden die Spielstrategien der Spieler dargestellt und die Summenanzahlen der Zellen in den jeweiligen Entscheidungen.
Verdeutlicht wird hier auch die wechselseitige Abhängigkeit beider Spieler: Wenn sich der Spieler $A$ (Marvin) für eine Spielstrategie entscheidet, so betrifft das auch Spielerin $B$ (Susanna) und die Auszahlung an beide Spieler.

Das Gefangendilemma ist in mehrerer Hinsicht ein Sonderfall, der sich in eine allgemeine Typologie von Spielen einsortieren lässt:

\begin{enumerate}
	\item Zum einem kann es \emph{Nullsummenspiele} geben.
	Dabei haben alle Spielausgänge die gleiche Summen.
	Es gibt lediglich Verteilungseffekte.
	\item Zum anderen kann es \emph{Positivsummenspiele} (wie oben dargestellt) geben.
	In einem Positivsummenspiel haben die Zellen unterschiedliche Summen. Unterschiedliche Spielausgänge führen also zu Wohlfahrtsverlusten oder -gewinnen.
	Tatsächliche Kooperation --- also wechselseitig vorteilhafte Zusammenarbeit --- ist nur bei Positivsummenspielen gegeben.
	Demnach geben Nullsummenspiele keine Aussage über menschliche Kooperation, da die Summenanzahl bei jedem Spielausgang gleich bleibt.
	Innerhalb von Positivsummenspielen kann ferner unterschieden werden zwischen:
	\begin{enumerate}
		\item Spiele \emph{totaler Harmonie}: \emph{Zellen} von Nash-Gleichgewicht und Wohlfahrtsoptimum fallen hier zusammen.
		Das Nash-Gleichgewicht liegt im Wohlfahrtsoptimum.
		Beispielhaft ist für diesen Spielausgang ist der Handel.
		Adam \textcite{Smith-1776-lq} Theorie der Handelsgewinne kann als Spiel totaler Harmonie verstanden werden:
		\begin{quote}
			``Wer sein eigenes Interesse verfolgt, befördert das Wohl der Gesamtgesellschaft häufig wirkungsvoller, als wenn er wirklich beabsichtigt, es zu fördern.
			Ich habe nie erlebt, dass viel Gutes von denen erreicht wurde, die vorgaben, für das öffentliche Wohl zu handeln.''\\*
			\parencite{Smith-1776-lq}
		\end{quote}
%VK TODO: Seitenzahl einfügen und dann besser parencite

		\item Spiele mit \emph{Kooperationsproblemen}: Summenanzahl von Wohlfahrtsoptimum und Nash Gleichgewicht unterscheiden sich.
		Ein Beispiel wäre das der Nationalen CO2-Emissionen.
		Entscheidet sich ein Land dafür, weniger Umweltschutzmaßnahmen zu treffen, so profitiert es davon nur, solange die anderen Ländern nicht die gleiche Strategie wählen.
	\end{enumerate}
\end{enumerate}

Die verschiedenen Spiele lassen sich auch in einem Baumdiagramm darstellen, wie in Abbildung \ref{fig:gefangenendilemma}.

\begin{dsafigure}
	\begin{center}
	\includegraphics[width=0.9\columnwidth]{img/summenspiele.jpg}
	\caption{Summenspiele nach \cite{Kleinberg-2009-oz}}
	\label{fig:gefangenendilemma}
	\end{center}
\end{dsafigure}


\paragraph{Die Spielstrategien}

Eine Strategie gilt als \emph{beste Antwort}, wenn sie zu der Strategie einer anderen Spielerin am besten passt, d.h. die eigene Auszahlung maximiert.
Hier sind mehrere beste Antworten möglich, wenn die Auszahlungen bei mehreren ``Antwort-Strategien'' gleich sind \parencite[153]{Kleinberg-2009-oz}.

Eine Strategie ist \emph{streng dominant}, wenn die Spielerin stets, unabhängig von den Mitspielern, die beste Auszahlung darstellt \parencite[vgl.][164]{Kleinberg-2009-oz}.


\paragraph{Die Spielausgänge}

Ein \emph{Nash-Gleichgewicht} entsteht, wenn die beiden Spieler Strategien gewählt haben, die jeweils die beste Antworten aufeinander sind.
Das \emph{soziale Wohlfahrtsoptimum} ist die Zellenkombination mit der höchsten Summe.

Der Erfolg einer Person im Spiel liegt somit nicht nur in seinen eigenen Entscheidungen, sondern darin welche Spielentscheidungen von allen anderen getroffen werden \parencite[vgl.][156]{Kleinberg-2009-oz}.


\paragraph{Die Axiome der Spieltheorie}

Nach Annahme der Axiome der Spieltheorie entscheidet sich die Spielerin in ihrem Handeln stets streng ökonomisch.
Das heißt, sie verfolgt die Strategie mit dem größtmöglichen Gewinn.
Außerdem wird angenommen, dass jeder den \emph{Spielplan} kennt und somit auch alle Spielstrategien und Mitspieler.
Die letzte Annahme basiert auf der rationalen, individuellen Nutzenmaximierung \parencite[vgl.][159]{Kleinberg-2009-oz}.


\paragraph{Das Gefangenendilemma lösen}

Außer durch den Einfluss eines Gewaltmonopolists oder einer Änderung der Axiome, wie z.B.\ der Annahme des ``Gemeinwohls'' (vgl. \citeauthor{rousseau-1762}), kann das Gefangenendilemma nicht gelöst werden, da alle Spieler nur aus Eigeninteresse handeln.
Dieses Problem wäre durch Tillys Theorie der \emph{Staatsgenese} gelöst, sie stellt hierzu einen Gewaltmonopolisten bereit \parencite{Tilly-1985-aa}.

Adam Smith lag somit mit seiner Annahme, dass die \emph{streng dominante} Spielstrategie stets auch am besten zum Allgemeinwohl beiträgt falsch, da man nicht grundsätzlich von Spielen mit \emph{besten Antworten} ausgehen kann und es dementsprechend, wie aufgezeigt, auch zu Kooperationsproblemen in menschlicher Interaktion kommen kann.


\paragraph{Anwendung der Spieltheorie}

Wie lässt sich die Spieltheorie in den Kurszusammenhang einordnen?
Das Modell der Spieltheorie stellt eine deutlich präzisere Formulierung des \emph{Kooperationsproblems} in der menschlichen Interaktion dar, indem es dieses auf ein mathematisches Modell zurückführt.
Durch diese Vereinfachung steht die Spieltheorie von Kleinberg aber auch in einem starken Kontrast zu dem Menschenbild der anderen Sozialwissenschaftler und Pädagogen, die wir im Kurs besprochen haben.
Kritisch zu hinterfragen ist, ob es sinnvoll ist, menschliche Kooperationsprobleme auf ein mathematisches System zurückzuführen beziehungsweise den Menschen durch Mathematik erklären zu wollen.