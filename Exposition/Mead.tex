%!TEX root=./Emile.tex

\subsection[Mead]{Der Symbolischer Interaktionismus nach Mead}

\epigraph{
	``Der Mensch wird am du zum ich.''
	\emph{Martin Buber}
	%MH TODO Für VK: Zitation?
}

\citeauthor{mead-1934en} beschäftigte sich Anfang des 20.\ Jahrhunderts damit wie Menschen interagieren und kooperieren.
Seine Theorie wird unter dem Begriff \emph{Symbolischer Interaktionismus} bekannt, also die Kommunikation zwischen Menschengruppen.
Daraus lassen sich sowohl soziologische als auch pädagogische Rückschlüsse ziehen.
Ein wichtiger Bestandteil des Werkes von Mead sind seine Ansichten zur Identitätsentwicklung, die aus einer Interaktion zwischen Individuum und Umwelt entsteht, genauer zwischen den Komponenten \emph{I}, \emph{Me} und \emph{Self}.
Dabei werden Wünsche und Ziele beziehungsweise sinnliche und körperliche Bedürfnisse durch das \emph{I} ausgedrückt.

Das \emph{Me} steht dem \emph{I} gegenüber und ist die Vorstellung eines Menschen, wie er von der Außenwelt betrachtet wird.
Es ist somit das reflektierte Ich.
Ein Beispiel dafür bietet der Fußball:

\begin{quote}
	``Am liebsten möchte ich alleine auf das Tor laufen und versuchen ein Tor zu schießen.''
\end{quote}

Hier spricht das \emph{I}, der Egoismus.

\begin{quote}
	``Doch ich weiß genau, wenn es schief geht, werden meine Mitspieler und mein Trainer sauer auf mich sein.''
\end{quote}

Diese Gedanken entsprächen dem \emph{Me}, das die Außenwelt wahrnimmt.
Das \emph{Self} wägt zwischen beiden Komponenten ab und macht die Persönlichkeit des Individuums aus.
Des Weiteren ist das \emph{Me} ständig in Bewegung, weil immer neue Ansichten, wie jemand von der Gesellschaft gesehen wird, im Verlauf des Lebens hinzukommen:
``Das `Me' ist die organisierte Gruppe von Haltungen anderer, die man selbst einnimmt.'' \parencite[218]{mead-1934en}.
Die Bildung des \emph{Me} ist nie wirklich abgeschlossen.
Aufgrund dessen, dass das \emph{Me} die Moralität beinhaltet und den Pluralismus der Gesellschaft umfasst, ist es der ständige Gegenspieler des \emph{I}.
Das \emph{I} wird auch impulsives Ich genannt, da es immer Neues und Schöpferisches in jede Situation bringt.
So schreibt Mead:
``Das \emph{I} liefert das Gefühl der Freiheit und Initiative'' \citeyear[221, Hervorhebung eingefügt]{mead-1934en}.

Den Zusammenhang zwischen \emph{I} und \emph{Me} beschreibt \citeauthor{mead-1934en} wie folgt:

\begin{quote}
	``Das \emph{I} ruft das \emph{Me} nicht nur hervor, es reagiert auch darauf.
	Zusammen bilden sie eine Persönlichkeit, wie sie in der gesellschaftlichen Erfahrungen erscheint.''\\*
	\textcite[221]{mead-1934en}.
\end{quote}

Die dritte Komponente ist das Ergebnis aus den ersten beiden Komponenten: das \emph{Self}.
Dies ist quasi die Identität, welche aus der ständigen Vermittlung zwischen \emph{I} und \emph{Me} entsteht.
Eine gelungene Identität ist erreicht, wenn \emph{I} und \emph{Me} in einer gleichgewichtigen Spannung zueinander stehen:
weder die totale Anpassung an die Erwartungen der Anderen, noch ein rein spontanes, impulsives Verhalten, das nur den eigenen Bedürfnissen entspricht.
Unser Kursthema wird hier also durch eine tiefgreifende interaktionistische Perspektive ergänzt, die sogar die Persönlichkeitsentwicklung aus Interaktion heraus erklärt.
Das \emph{Self} ist ein dynamisches Konstrukt.
Dynamisch ist es deshalb, weil \emph{Me} und \emph{I} durch die Interaktion mit Anderen in ständiger und ein Leben lang Bewegung sind.

Wie kann der Mensch nun eine individuelle, persönliche Identität aufbauen, wenn dieser Prozess von anderen Menschen durch das \emph{Me} abhängig ist?
Obwohl die Gesellschaft Erwartungen an jedes Individuum stellt, ist immer noch das \emph{I} vorhanden, welches vollkommen individuell ist und auf welches die Gesellschaft keinen Einfluss nehmen kann.
Des Weiteren sind die Erwartungen der Gesellschaft nicht bei jedem Menschen gleich, sondern auch in ihnen ist eine gewisse Individualität vorhanden.
Dies ist damit zu begründen, dass nicht an jeden Menschen in einer Gesellschaft die gleichen Erwartungen gestellt werden.
Die Werte und Normen, die ein Mensch im Verlauf seines Lebens, also während der Sozialisation, vermittelt bekommt, sind auch nicht bei jedem gleich, sondern individuell und kulturell geprägt.
Natürlich darf man nicht vergessen, dass dieser kleine Aspekt der Individualität, welcher im \emph{Me} enthalten, kaum Einfluss auf die eigentliche Individualität, welche ein jeder Mensch aufbaut.
Der Großteil der Individualität wird durch das \emph{I} hervorgerufen.

Zusammenfassend ist zu bemerken, dass die Gesellschaft und das Individuum nicht zu trennen sind, genauso wie das \emph{I} und das \emph{Me}.
Beides gehört zu einer Person und die genaue Individualität wird dadurch erreicht, wie diese Person mit der eigenen Individualität und den Normen und Werten der Gesellschaft umgeht.
