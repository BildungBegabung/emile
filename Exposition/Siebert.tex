%!TEX root=./Emile.tex
\subsection{Siebert: Der völlig autonome Mensch -- Lernen als Konstruktion von Wirklichkeit}

	\epigraph{
		``Der Mensch hat keinen Zugriff auf die objektive Realität''
			\emph{Heinz von Foerster}
	}

Die zentrale Frage einer konstruktivistischen Lerntheorie lautet:
Wie nimmt der Mensch die Welt wahr und inwiefern prägen ihn dabei individuelle Erfahrungen?
Bezieht man sich auf das einführende Zitat der Unmöglichkeit einer objektiven Wahrnehmung, so wird der Denkansatz des Konstruktivisten deutlich.
Laut Sieberts Theorie des Konstruktivismus ist die Wahrnehmung ``keine Abbildung der Außenwelt, sondern eine funktionale viable Konstruktion'' \parencite[6]{siebert-2003}.
Ausgehend von dieser Annahme stellt \citeauthor{siebert-2003}  die Kernthese auf, dass ``Menschen [...] autopoietische selbstreferenzielle, operational geschlossene Systeme [sind].'' \parencite[5]{siebert-2003}.
Damit geht von einen autonomen Menschen aus, der sich durch Selbstorganisation auszeichnet (\emph{Autopoesie}).

Die Lebensauffassung ändert sich demnach permanent aufgrund von Erfahrungen und Ereignissen, die den Menschen betreffen.
Der Konstruktivismus geht bei seiner Theorie von einem interpretativen \emph{Paradigma}, der \emph{Sichtweise} des Individuums, aus.
Der Mensch entscheidet als \emph{geschlossenes System}, ob die verfügbaren Informationen für ihn lebensdienlich (\emph{viabel}) sind.
Abhängig von den individuellen Determinationen, mit denen ein Mensch im Leben konfrontiert wird, verändert sich so seine Wahrnehmung der Umwelt.
Der Mensch ist demnach strukturdeterminiert \parencite[vgl.][5]{siebert-2003}.
Aufbauend auf seinem individuell geprägten Menschenbild geht \citeauthor{siebert-2003}  davon aus, dass ein Wechselverhältnis zwischen der Wahrnehmung des Menschen und der Umwelt existiert.
Die individuelle Wahrnehmung wird von Viabilität und Rekursivität bestimmt, womit ein Ansatz zur Lerntheorie gefunden werden kann.

\paragraph{Konstruktivismus in der Pädagogik}

Geht man von der vollkommenen Subjektivität eines Individuums aus, so ist das auf allgemeine Lernmethoden beruhenden Schulsystem nicht in der Lage zu \emph{lehren}.
Demzufolge schlussfolgert \citeauthor{siebert-2003}, dass ein \emph{Bildungskanon"} nicht der menschlichen Natur des Lernens entspricht \parencite[vgl.][28]{siebert-2003}.
In einer pluralisierten Gesellschaft, ``die durch fortschreitende \emph{Individualisierung} [...] gekennzeichnet ist, erscheint auch eine \emph{Pluralisierung des Bildungsbegriffs} angemessen'' \parencite[vgl.][29]{siebert-2003}.

Wie Wissen am besten vermittelt wird und so der Lernerfolg maximiert werden kann, bleibt eine grundlegende Frage in der Schule -- sowie in der Lerntheorie.
\citeauthor{siebert-2003}  selbst geht auf diese Frage wie folgt ein:
``Menschen als selbstgesteuerte \emph{Systeme} können von der Umwelt nicht determiniert, sondern allenfalls \emph{perturbiert}, das heißt, **„gestört“** und \emph{angeregt} werden.'' \parencite[vgl.][5]{siebert-2003}.

Des Weiteren hängt die Auswahl der Informationen, also was und ob gelernt wird, von der direkten Relevanz der Informationen für das Individuum ab.
Lernen ist also in erster Linie ein eigenwilliger und eigensinniger Prozess und dient primär der Selbsterhaltung (\emph{Autopoiese}).

Aufgrund dieser Autopoiese entscheidet der Mensch (\emph{das System}) \emph{autonom} über die Verarbeitung der Inputs.
Der Mensch kann maximal durch Perturbationen (\emph{(An)Reize}) zum (Um)Denken angeregt werden.
Diese Möglichkeit der Perturbation kann der Lehrende zur Vermittlung von Wissen nutzen und damit gegebenenfalls neue Sichtweisen für den Lernenden eröffnen.

Die Perturbation kann sowohl auf lerntheoretischen Ansätze angewandt werden, als auch auf soziologische Problematiken übertragen werden.
Problematisch ist hierbei die vollkommene Isolierung des Einzelnen in der eigenen subjektiven Welt - die völlige Autonomie.
Diese angesprochene Isolation, besonders im radikalen Konstruktivismus, ist besonders gut an der Geschichte von \emph{Peter Bichsel} ``Ein Tisch ist ein Tisch'' veranschaulicht.
Hierbei wird die Situation eines Mannes beschrieben, der in Unzufriedenheit lebt.
Um sich aus dieser aussichtslosen Lebensweise zu befreien beginnt er bekannte Gegenstände mit anderen Namen zu versehen.
Demzufolge wurde der Stuhl zum Tisch oder das Bett zum Bild.
``'Jetzt ändert es sich', rief er, und er sagte von nun an zu dem Bett Bild.'' \parencite[vgl.][18]{bichsel-1990}.

Die radikale Änderung seiner eigenen Wahrnehmung und Kommunikation verursachte eine vollkommene Abgrenzung von jeglichen gesellschaftlichen Verbindungen, die Bichsel wie folgt beschreibt:
``Viel schlimmer war, sie konnten ihn nicht mehr verstehen. Und deshalb sagte er nichts mehr. Er schwieg, sprach nur noch mit sich selbst, grüßte nicht einmal mehr.'' \parencite[vgl.][25]{bichsel-1990}.

Der Verlust der Kooperation mit der Umwelt durch die Isolierung in der Subjektivität und damit auch die fehlende Beziehung mit Menschen, zeigt die Grenzen des (radikalen) Ansatzes der konstruktivistischen Theorie auf.
Diese Tatsache impliziert ein Kooperationsproblem zwischen koexistierenden \emph{Individuen}, da jene \emph{Realitäten} sich ausgehend vom Menschen unterscheiden.
Völlige Autonomie und damit die Abgrenzung von jeglichen Richtlinien ist somit ein essentielles Problem des Zusammenlebens.
