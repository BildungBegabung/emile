%!TEX root=./Emile.tex
\section[Kursbeschreibung]{Aus der Kursbeschreibung}
\authors{Verena Kasztantowicz, Maximilian Held}

%VK: insgesamt kürzen für die Doku?
Der Kurs ``Schule und Demokratie --- Deliberative Politik und Inklusive Pädagogik'' (kurz: \emph{emile}) erarbeitet einen zentralen Konflikt aus Pädagogik und politischer Theorie:
den Widerspruch und die wechselseitige Bedingtheit von inhärenter Gleichwertigkeit und persönlicher Autonomie im menschlichen Zusammenleben bei geschätzter Differenz und beobachteter Ungleichheit \emph{zwischen} den Menschen.

Die \emph{Pädagogik} fragt:
Wenn alle Schülerinnen als \emph{gleich} --- im Sinne ihrer Bildsamkeit und ihrer Rechte --- angesehen werden \parencites[vgl.][]{UN-2008,benner-2012}.
jedoch etwa hinsichtlich ihrer körperlichen und geistigen Voraussetzungen \emph{verschieden} im Klassenzimmer agieren, wie kann die Schule sie gleichzeitig zum \emph{selbstbestimmten} und \emph{gemeinschaftlichen Leben} erziehen?

\begin{quote}
    ``Die Schule der Nation ist die Schule.''\\*
    (Willy Brandt 1969)
\end{quote}

Ähnlich fragt die \emph{politische Theorie}:
Wenn alle Bürger inhärent gleichwertig und persönlich autonom sein \emph{sollen} \parencite[etwa][]{Dahl-1989-aa}, allerdings etwa hinsichtlich ihrer kognitiven Fähigkeiten \parencite{Rosenberg-2002-aa}, ihres Wissens \parencite[etwa][]{Converse-1970-aa} oder ihrer Sprache unterschiedlich an der Wahlurne eintreffen, welche demokratischen Institutionen können dann die Selbstbestimmung unter Gleichen garantieren?

\begin{quote}
    ``A popular government, without popular information or the means of acquiring it, is but a prologue to a farce or a tragedy, or perhaps both.''\\*
    (James Madison 1822)
\end{quote}

Einen gemeinsamen Ausgangspunkt bietet der Sozialphilosoph Jean-Jacques Rousseau mit ``Émile oder über die Erziehung'' \parencite{rousseau1762} und dem ``Le Contract Social'' \parencite*{Rousseau-1762-b} die --- auch durch ihr Scheitern und ihre romantische Verklärung --- viele Debatten des 19.\ und 20.\ Jahrhunderts vorwegnehmen.
Aus der Pädagogik werden zum einen Bedingungen und Funktionen menschlicher Interaktion und individueller Lernprozesse betrachtet \parencites{siebert-2003,benner-2012,mead-1934en}.
Die Erkenntnisse daraus werden zum anderen mit historischen und aktuellen Vorschlägen schulischer Umsetzung verglichen: der reformpädagogischen Freinet-Schule und dem Konzept der Entschulung \parencites{Freinet1979,Illich-1971}.

Aus der Politikwissenschaft beinhaltet der Kurs einen Abriss der Staatsgenese \parencite{Tilly-1985-aa} sowie liberaler normativer Theorie \parencites{Dahl-1989-aa}.
Anschließend werden zentrale Widersprüche \parencite{Condorcet1785,Arrow1950} der aggregativen Demokratie beleuchtet, und --- teilweise als Antwort darauf --- eine alternative, deliberative Theorie der Demokratie in den Blick genommen \parencite{Cohen-1989-aa,Habermas1988a}.

Abschließend treffen sich die beiden Disziplinen wieder mit Deweys ``Democracy and Education'' \parencite{Dewey-1916} und versucht eine radikale Synthese der beiden Bezugsdisziplinen,
%VK haben wir nicht mehr geschafft:
%nämlich, dass die inklusive Schule eine demokratische Schule sein müsse und deliberative Demokratie gleichermaßen der operativen Metapher der Schule folgen solle \parencite{Rosenberg-2002-aa}.

Was der Kurs \emph{nicht} bietet:
Auch wenn der Titel ``Schule der Demokratie'' es vermuten lässt, geht es in diesem Kurs nicht darum, uns zu demokratischeren Menschen zu erziehen --- wenn überhaupt, dann ist die Frage eher umgekehrt:
Wie kann Schule, wie können Institutionen, Lernprozesse und letztlich auch Beziehungen demokratischer gestaltet werden?
Schließlich geht es in unserem Kurs auch nicht um das, was man gemeinhin unter Politik versteht --- vor allem keine Tages- oder Schulpolitik --- sondern allgemeiner darum, was eigentlich kollektiv verbindlichen Entscheidungen Legitimität verleihen könnte.


\paragraph{Kooperatives Schreiben}
%VK: Könntest du das noch überarbeiten ohne die Reflexion zu doppeln, MH?
Nicht nur inhaltlich, sondern auch in der Wahl der Arbeitsmittel ist der Kurs auf Kooperation hin ausgelegt.
Die Teilnehmenden erlernen und erproben den Umgang mit drei professionellen Programmen aus der Open-Source-Softwareentwicklung für das Schreiben von Texten in der Pädagogik und den Sozialwissenschaften:

\begin{enumerate}
	\item \emph{Git}, einem Versionskontroll- und Quellverwaltungsprogramm zur gemeinsamen Softwareentwicklung,
	\item \emph{Github.com}, einem kommerziellen Hoster von \emph{Git} und Anbieter eines darauf aufbauenden sozialen Netzwerks,
	\item \emph{Atom.io}, einem quelloffenen Software-Editor.
\end{enumerate}