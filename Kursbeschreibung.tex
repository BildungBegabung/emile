%!TEX root=./Emile.tex
\section[Kursbeschreibung]{Aus der Kursbeschreibung}
\authors{Verena Kasztantowicz, Maximilian Held}

%VK: insgesamt kürzen für die Doku?
Der Kurs ``Schule und Demokratie --- Deliberative Politik und Inklusive Pädagogik'' (kurz: \emph{Émile}) erarbeitet einen zentralen Konflikt aus Pädagogik und politischer Theorie:
den Widerspruch und die wechselseitige Bedingtheit von inhärenter Gleichwertigkeit und persönlicher Autonomie im menschlichen Zusammenleben bei geschätzter Differenz und beobachteter Ungleichheit \emph{zwischen} den Menschen.

Die \emph{Pädagogik} fragt:
Wenn alle Schülerinnen als \emph{gleich} --- im Sinne ihrer Bildsamkeit und ihrer Rechte --- angesehen werden \parencites[vgl.][]{UN-2008,benner-2012}.
jedoch etwa hinsichtlich ihrer körperlichen und geistigen Voraussetzungen \emph{verschieden} im Klassenzimmer agieren, wie kann die Schule sie gleichzeitig zum \emph{selbstbestimmten} und \emph{gemeinschaftlichen Leben} erziehen?

\begin{quote}
    ``Die Schule der Nation ist die Schule.''\\*
    (Willy Brandt 1969)
\end{quote}

Ähnlich fragt die \emph{politische Theorie}:
Wenn alle Bürger inhärent gleichwertig und persönlich autonom sein \emph{sollen} \parencite[etwa][]{Dahl-1989-aa}, allerdings etwa hinsichtlich ihrer kognitiven Fähigkeiten \parencite{Rosenberg-2002-aa}, ihres Wissens \parencite[etwa][]{Converse-1970-aa} oder ihrer Sprache unterschiedlich an der Wahlurne eintreffen, welche demokratischen Institutionen können dann die Selbstbestimmung unter Gleichen garantieren?

\begin{quote}
    ``A popular government, without popular information or the means of acquiring it, is but a prologue to a farce or a tragedy, or perhaps both.''\\*
    (James Madison 1822)
\end{quote}

Einen gemeinsamen Ausgangspunkt bietet der Sozialphilosoph Jean-Jacques Rousseau mit ``Émile oder über die Erziehung'' \parencite*{rousseau-1762} und dem ``Le Contract Social'' \parencite*{Rousseau-1762-b} die --- auch durch ihr Scheitern und ihre romantische Verklärung --- viele Debatten des 19.\ und 20.\ Jahrhunderts vorwegnehmen.
Aus der Pädagogik werden zum einen Bedingungen und Funktionen menschlicher Interaktion und individueller Lernprozesse betrachtet \parencites{siebert-2003,benner-2012,mead-1934en}.
Die Erkenntnisse daraus werden zum anderen mit historischen und aktuellen Vorschlägen schulischer Umsetzung verglichen: der reformpädagogischen Freinet-Schule und dem Konzept der Entschulung \parencites{Freinet1979,Illich-1971}.

Aus der Politikwissenschaft beinhaltet der Kurs einen Abriss der Staatsgenese \parencite{Tilly-1985-aa} sowie liberaler normativer Theorie \parencites{Dahl-1989-aa}.
Anschließend werden zentrale Widersprüche \parencite{Condorcet1785,Arrow1950} der aggregativen Demokratie beleuchtet und Dynamiken moderner Gesellschaft anhand von Netzwerk- und Spieltheorie in den Blick genommen \parencite{Kleinberg-2009-oz}.
Abschließend untersucht der Kurs mit \emph{Katallaxie} \parencite{hayek-1945} und \emph{kommunikativem Handeln} \parencite{Habermas-1998-aa} zwei diametrale wie radikale Reformvorschläge für diese krisenhafte Komplexität der politischen Gegenwart.

Schließlich treffen sich die beiden Disziplinen wieder mit Deweys ``Democracy and Education'' \parencite*{Dewey-1916} und versucht eine radikale Synthese der beiden Bezugsdisziplinen, nämlich, dass die Schule eine demokratische Schule sein müsse und deliberative Demokratie gleichermaßen der operativen Metapher der Schule folgen solle \parencite{Rosenberg-2002-aa}.
%VK haben wir nicht mehr geschafft:
	%MH: genauer wa fehlt ist sowohl die Praxis der Deliberation als auch die der Inklusion.

% Was der Kurs \emph{nicht} bietet:
% Auch wenn der Titel ``Schule der Demokratie'' es vermuten lässt, geht es in diesem Kurs nicht darum, uns zu demokratischeren Menschen zu erziehen --- wenn überhaupt, dann ist die Frage eher umgekehrt:
% Wie kann Schule, wie können Institutionen, Lernprozesse und letztlich auch Beziehungen demokratischer gestaltet werden?
% Schließlich geht es in unserem Kurs auch nicht um das, was man gemeinhin unter Politik versteht --- vor allem keine Tages- oder Schulpolitik --- sondern allgemeiner darum, was eigentlich kollektiv verbindlichen Entscheidungen Legitimität verleihen könnte.


\paragraph{Kooperatives Schreiben}

Die den Kursinhalt leitenden Herausforderungen menschlicher Kooperation nimmt \emph{Émile} auch in der gemeinsamen Schreibarbeit an.
Die Teilnehmenden erlernen aktuelle Praktiken und Programme aus der Open-Source-Softwareentwicklung und schreiben \emph{selbstorganisiert} \emph{einen gemeinsamen} Text, der in dieser Dokumentation vorliegt.