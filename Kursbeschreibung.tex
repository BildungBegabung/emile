\section[Kursbeschreibung]{Aus der Kursbeschreibung}

Der Kurs ``Schule der Demokratie --- Deliberative Politik und Inklusive Pädagogik'' (kurz: \lstinline{emile}) erarbeitet einen zentralen Konflikt aus Pädagogik und politischer Theorie:
den Widerspruch und die wechselseitige Bedingtheit von inhärenter Gleichheit und persönlicher Autonomie im menschlichen Zusammenleben.

Die \emph{Pädagogik} fragt:
Wenn alle Schülerinnen als \emph{gleich} --- im Sinne ihrer Bildsamkeit und ihrer Rechte --- angesehen werden (vgl.\ Benner 2012; UN-Behindertenrechtskonvention 2006),
%MH FIXME korrekte Zitation einfügen
jedoch etwa hinsichtlich ihrer körperlichen und geistigen Voraussetzungen \emph{verschieden} im Klassenzimmer agieren, wie kann die Schule sie gleichzeitig zum \emph{selbstbestimmten} und \emph{gemeinschaftlichen Leben} erziehen?

\begin{quote}
    "’Die Schule der Nation ist die Schule.''\\*
    Willy Brandt (1969)
\end{quote}

Ähnlich fragt die \emph{politische Theorie}:
Wenn alle Bürger inhärent gleich und persönlich autonom sein \emph{sollen} (etwa Dahl 1989), allerdings etwa hinsichtlich ihrer kognitiven Fähigkeiten (Rosenberg 2007), ihres Wissens (etwa Converse 1970) oder ihrer Sprache unterschiedlich an der Wahlurne eintreffen, welche demokratischen Institutionen können dann die Selbstbestimmung unter Gleichen garantieren?
%MH TODO richtige Zitationen einfügen

\begin{quote}
    ``A popular government, without popular information or the means of acquiring it, is but a prologue to a farce or a tragedy, or perhaps both.''\\*
    James Madison (1822)
\end{quote}

Einen gemeinsamen Ausgangspunkt bietet der Sozialphilosoph Jean-Jacques Rousseau mit ``Émile oder über die Erziehung'' (1762) und dem ``Contract Social'' (1764) die --- auch durch ihr Scheitern und ihre romantische Verklärung --- viele Debatten des 19.\~ und 20.\~ Jahrhunderts vorwegnehmen.
Aus der Pädagogik werden zum einen mithilfe von Lern- und Entwicklungstheorien Bedingungen und Funktionen menschlicher Kooperation und individueller Lernprozesse betrachtet (etwa Wygotski, Piaget, Reich).
%MH FIXME: Zitationen
Die Erkenntnisse daraus werden zum Anderen mit historischen und aktuellen Vorschlägen schulischer Umsetzung verglichen, etwa mit der reformpädagogischen Freinet-Schule, dem Konzept der Entschulung und der Inklusions- bzw. Integrationspädagogik (Freinet 1964, Illich 1971, Eberwein 2009).
%MH FIXME: Zitationen

Aus der Politikwissenschaft beinhaltet der Kurs einen Abriss der Staatsgenese (Tilly 1984, Marshall 1980) sowie liberaler normativer Theorie (Rawls, Dahl).
%MH FIXME: Zitationen
Anschließend werden zentrale Widersprüche (Condorcet, Arrow) und Dysfunktionen (Tsebelis) der aggregativen Demokratie beleuchtet, und --- teilweise als Antwort darauf --- eine alternative, deliberative Theorie der Demokratie in den Blick genommen (Cohen 1989).
%MH FIXME: Zitationen

Abschließend treffen sich die beiden Disziplinen wieder mit Deweys ``Democracy and Education'' (1916) und versucht eine radikale Synthese der beiden Bezugsdisziplinen, nämlich, dass die inklusive Schule eine demokratische Schule sein müsse (Feuser 2001) und deliberative Demokratie gleichermaßen der operativen Metapher der Schule folgen solle (Rosenberg 2002).
%MH FIXME: Zitationen

Was der Kurs *nicht* bietet:
Auch wenn der Titel ``Schule der Demokratie'' es vermuten lässt, geht es in diesem Kurs nicht darum, uns zu demokratischeren Menschen zu erziehen --- wenn überhaupt, dann ist die Frage eher umgekehrt:
Wie kann Schule, wie können Institutionen, Lernprozesse und letztlich auch Beziehungen demokratischer gestaltet werden? Schließlich geht es in unserem Kurs auch nicht um das, was man gemeinhin unter Politik versteht --- vor allem keine Tages- oder Schulpolitik --- sondern allgemeiner darum, was eigentlich kollektiv verbindlichen Entscheidungen Legitimität verleihen könnte.


\paragraph{Anforderungen}

Teilnehmende sollten bereit sein, (zum Teil englische und französische) Fach- und Primärtexte zu erarbeiten, professionelle Computerprogramme zu erlernen und kooperativ damit zu arbeiten (Git) und Lust auf kontroverse, aber stets theoriegeleitete Diskussionen mitbringen.
