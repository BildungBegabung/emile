\subsection{Freinet - Neue Blickwinkel durch die Reformpädagogik auf eine Schule der Demokratie?}

	\epigraph{
		``Am Anfang jeder Eroberung steht nicht das abstrakte Wissen, sondern die Erfahrung, die Übung und die Arbeit.''}
	{
		\emph{Célestin Freinet}
	}

\subsubsection{Freinets didaktische Konzeption einer Schule als Lösung des Kooperationsproblems}

	\begin{quote} ``Es geht darum, unser ganzes Erziehungssystem von der materiellen Basis her umzugestalten.''
		\cite[S.~99]{Freinet1979}
	\end{quote}

Diese These Freinets scheint zunächst überraschend.
Für gewöhnlich beginnt man eine solche Revolution mit einem theoretischen Konzept und nicht mit einer Handvoll Lernkarteien in einem umgeräumten Klassenzimmer.
Vielleicht ist aber gerade dieser, auf den ersten Blick etwas unorthodoxe Blick auf die Pädagogik notwendig, um die Grundfrage zwischen persönlicher Autonomie und inhärenter Gleichheit zu beantworten?

Für Freinet ist es notwendig, zuerst gute und funktionstüchtige Werkzeuge zu haben, da ohne diese eine erfolgreiche Anwendung seiner reformpädagogischen Ideen nicht möglich wäre.
Wenn die Lehrerin nicht auf *ausgewähltes Material* zurückgreifen könnte, wären die Durchführung bestimmter Experimente gefährlich und ``die Schülerinnen würden infolge ihres technischen Unvermögens entmutigt.''\citep[S.~98]{Freinet1979}.
Weiterhin ist die richtige Organisation dieser Materialien notwendig, ``um disziplinarische Probleme einer Klasse [zu lösen]'' (ebd.).
Erst in einer materialistisch gut ausgestatteten und überdacht organisierten Schule kann man sich also an die Umsetzung der Ideen wagen.

Seine grundlegenden pädagogischen Ideen hält \citeauthor{Freinet-2000a} in 30 sogenannten Invarianten fest.
Invarianten sind unveränderliche Wahrheiten \citep[vgl.][S.~488]{Freinet-2000a}.
In ihnen zeigt sich, wie bedeutsam die \emph{intersubjektive Beziehung} zwischen Lehrerinnen-Schülerinnen ist.
Freinet ist der Meinung, das Verhältnis zwischen Lehrerin und Schülerin solle so sein wie das \emph{Verhältnis einer Mutter zu ihrem Kind}.
Er sagt, ein Lehrer müsse seinen ``Schülern das gleiche Vertrauen schenken können, das eine Mutter ihrem kleinen Kinde schenkt'' \citep[105]{Freinet1979}.
Das beinhaltet ebenfalls ``größtmögliche Ehrlichkeit'' dem Kind gegenüber (ebd. S. 110).
Darüber hinaus ist Freinet davon überzeugt, dass Kinder am besten individuell und - soweit möglich - selbstgesteuert lernen sollten.
Für Freinet ist ``die ständige Anwesenheit der Lehrerin nirgendwo erforderlich'' \citep[105]{Freinet1979}, wovon sich ableiten lässt, dass der Lehrer nur beratend und unterstützend wirken darf und soll.
Allerdings darf selbst diese Hilfe erst dann genutzt werden, wenn das Kind die Lehrerin von sich aus fragt.
Wichtig ist ebenfalls das empathische Auftreten der Lehrersin.
Zwar erkennt Freinet an, ``dass es doch meistens nicht in ihrer Macht steht zu einer [...] liebevollen Begegnung mit dem Schüler zu gelangen'' \citep[101]{Freinet1979}, dennoch darf der Lehrer nicht drohen, schimpfen oder strafen (vgl. ebd. S. 103).
Vielmehr gibt er ``behutsam einige Hinweise'' (ebd.).
Diese Lehrerrolle räumt dem Kind einen großen Rahmen der Selbstbestimmung ein.
Freinet ist der Überzeugung, dass ``zu einer Arbeit gezwungen zu werden [...] lähmt'' \citep[vgl.][S.~495]{Freinet-2000a}.
Dies führt dazu, dass die \emph{kindliche Partizipation} ein entscheidender Teil von Freinets didaktischer Konzeption ist und zeigt darüber hinaus, dass Freinet auf persönliche Autonomie einen sehr hohen Wert legt.
Wenn Freinet schreibt, ``dass zwar der Einzelne in Funktion der Gemeinschaft arbeitet, dass aber die Gemeinschaft Einzel-, Partner- und Gruppenleistungen unbedingt anzuerkennen hat'' \citep[87]{Freinet1979} wird sein Verhältnis zu inhärenter Gleichheit deutlich:
Diese ist durchaus ein Teil seiner pädagogischen Vorstellungen, aber sieht er es als viel entscheidender an, dass innerhalb einer solchen Gruppe gleichwertiger Menschen die persönliche Autonomie gewahrt bleibt.

Darüber hinaus sieht \citeauthor{Freinet1979} auch keine Notwendigkeit eines verallgemeinernden Lehrplanes.
Er ist sich sicher, ``Was das Kind nicht heute lernt, nicht in dieser Woche, ja selbst nicht in diesem Jahre lernt, dass wird es sich später aneignen'' \citep[~101]{Freinet1979}.
Es ist also wichtiger, ``dass das Kind seine Individualität entwickelt'' (ebd. S. 105), als dass es spezielle Lerninhalte eines vorgegebenen Curriculums erfasst.
Bei dieser Art der Pädagogik ist, da es keine Vorschriften gibt, das individuelle Interesse sehr wichtig.
Diese Ablehnung eines Bildungskanons ist ein weiteres Plädoyer für persönliche Autonomie, da sie zeigt, wie sehr Freinet jegliche Gleichstellung der Edukanten in einem Bildungssystem ablehnt.
Er ist der Auffassung, dass die notwendigsten Dinge von sich aus durch das Kind irgendwann gelernt werden.
Dies kann man nur nicht auf einen bestimmten Zeitraum festlegen.

Freinet unterscheidet hierbei zwischen Interessenkomplexen und Interessenzentren.
Die \emph{Interessenkomplexe (Complexe d'intérêts)} gehen von der Schülerin aus, d.h. sie hat gewisse Interessen, welche sie erarbeiten möchte.
Ausgehend von diesen Interessenkomplexen strukturiert sich die gesamte pädagogische Arbeit.
Für Freinet muss zuallererst die Schülerin seine Ambitionen für einen bestimmten Bereich zeigen, wovon anschließend der Lehrer behilflich sein kann, das Wissen zu ordnen.
Freinet schreibt, die Lehrerin muss den Kindern ``bei der manuellen, künstlerischen und geistigen Verwirklichung ihrer vorherrschenden Möglichkeiten'' \citep[~90]{Freinet1979} helfen.
Im Gegensatz dazu, werden \emph{Interessenzentren (Centres d'intérêts)} vom Lehrer inhaltlich vorstrukturiert und gehen nicht von der Schülerin aus.
Freinet spricht daher von der ``Überlegenheit des Interessenkomplexes [...] gegenüber den Interessenzentren'' \citep[~89]{Freinet1979}.
Gerade die Überlegenheit der Interessenkomplexe macht deutlich, dass persönliche Autonomie der Grundpfeiler der Freinetschule ist.
Schließlich hat jedes Kind seine eigenen individuellen Interessenkomplexe, welche auch individuell betrachtet werden sollen.

\subsubsection{Bezüge zu anderen Theoretikern}

\paragraph{Bezüge zum pädagogischen Konstruktivismus nach Siebert}

Freinet geht davon aus, dass ein Edukant am besten auf seine ganz individuelle Art und Weise lernt und sich die Welt erklärt \citep[vgl.][~496]{Freinet-2000a}.
Daraus lässt sich ableiten, dass in der Individualität für Freinet ein großer und wichtiger Vorteil liegt.
Es lässt sich also schlussfolgern, dass Freinet einem konstruktivistischem Lernkonzept wie dem von Horst Siebert zustimmen würde.
Dieser behauptet: ``Die [...] so erzeugte Wirklichkeit ist keine Repräsentation [...] der Außenwelt, sondern eine funktionale, viable Konstruktion, die von anderen Menschen geteilt wird und die sich biographisch und gattungsgeschichtlich als lebensdienlich erwiesen hat'' \citep[~6]{siebert-2003}.

Allerdings würde Freinet Sieberts Vorstellung des radikalen Konstruktivismus ablehnen, da dieser Kommunikation unter Menschen keinen Platz einräumt.
Es würden danach sämtliche Gedanken nur im Kopf eines Menschen rekursiv rekonstruiert werden \citep[vgl.][~10]{siebert-2003} und somit hätte eine Schule wie sich Freinet sie vorstellt keine Berechtigung.
Viel eher würde Freinet dem sozialen Konstruktivismus zustimmen.
Dieser setzt besonders auf soziale Interaktion einen großen Wert.

\paragraph{Bezüge zu den Prinzipien pädagogischen Denkens und Handelns nach Benner}

Benner sieht in einem Erziehungsprozess immer eine Aufforderung zur Bildsamkeit beziehungsweise zur Selbsttätigkeit.
Er bezieht sich dabei auf Johann Friedrich Hebart:
``Der Grundbegriff der Pädagogik ist die Bildsamkeit'' \citep[~70]{benner-2012}.

Er räumt damit dem Individuum eine große Menge an Autonomie ein.
Gerade diese Autonomie, die durch Selbsttätigkeit entsteht, ist ein Grundsatz in Freinets didaktischer Konzeption.
Hilfe gibt Freinet nur dann, wenn der Edukant diese auch benötigt, dies kommt Benners Aufforderung zur Selbsttätigkeit sehr nahe, die nur dann notwendig ist, wenn der Edukant nicht dazu fähig ist sich selbst zu bilden \citep[vgl.][~91]{benner-2012}.

\paragraph{Bezüge zum symbolischen Interaktionismus nach Mead}

Meads symbloischer Interaktionismus legt durch Kommunikation den Grundstein für die menschliche Identität.
Diesen Ansatz würde Freinet unterstützen, da seine Didaktik ähnlich abläuft ``wie in der Familie'' \citep[~109]{Freinet1979}(.
Somit ist Kommunikation sehr wichtig.

\paragraph{Bezüge zur Inklusiven Schule nach Zimpel}

Im Hyperzyklus nach \citeauthor{Zimpel2012} ist jedes Individuum wichtig für die Arbeit in einer Gruppe.
Fällt auch nur ein Individuum weg, so ist die gesamte Struktur des Systems gefährdet und es kann nicht mehr jeder profitieren.
Für Freinet ergibt sich diese Form des Lernens nochx nicht.
Er stellt besonders die Beziehung zwischen Lehrer und Schüler in den Vordergrund und sieht den Fokus der Arbeit eines
Unter den Schülern selbst herrscht jedoch ein großes Maß an Autonomie.
Man erkennt jedoch, das bereits die Tendenz zu einer Schule mit solch inklusiver Arbeit vorhanden ist.
