\subsection{Hayek}

\epigraph{
		``The curious task of economics is to demonstrate to men how little they really know about what they imagine they can design.''}
	{
		\emph{Friedrich A. Hayek
	}

In der Schule werden Informationen vermittelt, welche Informationen lohnt es sich zu vermittelt?
In der Demokratie müssen wir kollektiv verbindliche Entscheidungen treffen, indem wir Information zusammentragen.
Doch wie sehen diese Informationen aus? Welche Arten von Infos gibt es?

Wissen ist \citeauthor{hayek-1945} zufolge der Beweggrund wirtschaftlichen Handlens.
In seinem Buch ``The use of knowledge in society'' behandelt er die Frage wie ein wirtschaftliches System aussehen sollte.
Informationen spielen dabei eine große Rolle, da diese für Kollektiventscheidungen unerlässlich sind, um demokratische Entscheidungen fällen zu können.
Dabei betrachtet Hayek das Individuum ontologisch als kleinste Einheit, und fragt davon ausgehend wie ein wirtschaftliches System ausgerichtet sein sollte.
Wissen jedoch ``never exists in concentrated or integrated form, but solely as the dispersed bits of incomplete and frequently contradicitionary knowlegde which all the seperate individuals possess'' \citep[520]{hayek-1945}.

Nach \citeauthor{hayek-1945} entstehen manche Informationen erst in der konkreten Problemsituation, etwa durch eine Abwägung von verschiedenen Gütern bei gegebenen Preissignalen.
Ein Beispiel aus dem Akademie-Alltag: Oft treffen die Mengen an Kaffee und Tee nicht die tatsächliche Nachfrage und es gibt Engpässe oder Überangebote.
\citeauthor{hayek-1945} wäre vielleicht skeptisch ob der Chancen jedweder zentraler Planung – etwa durch die Akademieleitung; schließlich würde das eine effektive Abfrage von lokaler Teilnehmerinnen-Information über Kaffee- und Teegeschmack erfordern.
Plausiblerweise würden die Teilnehmerinnen ihre zukünftigen Bedürfnisse ungenau einschätzen oder die Küche würde Nachfragespitzen nicht antizipieren.
Eine katallaktische Lösung im Sinne Hayeks könnte es sein, anstelle des pauschal und subventioniert bereitgestellten Kaffees, Heißgetränke über einen Markt bereit zu stellen.
Möglicherweise hätten dann kaufende Teilnehmerinnen Zugang zu ihren tatsächlich, am Markt bepreistem Bedürfnis an Heißgetränken und die Anbieter hätten Anreize Informationen über zu erwartende Nachfragespitzen (etwa in der Doku-nacht) einzuholen.

\citeauthor{hayek-1945} schätzt charakteristisch das Preissystem als geeigneter ein um lokale, und kontingente – also abhängig von Umständen – Informationen zu sammeln.

Wissen besteht nicht nur aus \emph{scientific knowledge}, immanentem Faktenwissen,  Wissen entsteht oft auch lokal, also erst in einer gegebenen Situation.
Dies ist das Argument das \citeauthor{hayek-1945} einer Planwirtschaft entgegenstellt:
Planwirtschaft benötigt eine zentrale Aggregation von Informationen; diese ist aber oft ineffizient oder mangelhaft.

Es ist unwirtschaftlich Informationen zusammenzutragen, die in einer konkreten Situation jedoch nicht einmal zutreffen.
\citeauthor{hayek-1945} zieht ein freiheitlicheres System der Marktwirtschaft aus Gründen der Effizienz vor.
Hayeks Argument für freiheitlichen Markt besteht in seiner Annahme, dass jeder der Kompetenteste für sich selber sei.

	``Every individual [...] posseses unique information of which beneficial use can be made only if the decisions depending on it are left to him or are made with his active cooperation.'' \citep[521]{Hayek-1945}

Ähnlich forderte \citeauthor{Dahl-1989-aa} auch nach seiner Annahme der persönliche Autonomie als Grundlage einer Demokratischen Gesellschaft, wenn er formuliert,
``Every adult is the best judge of his or herself interest ''\citep[100]{Dahl-1989-aa}.

Für \citeauthor{Dahl-1989-aa} ergibt sich daraus eine politische Dimension, die Demokratie, in der jeder, zum Beispiel durch Wahlen, für sich selbst entscheiden kann.
\citeauthor{hayek-1945}  sieht vor allem die wirtschaftliche Dimension:`` weniger Regierung'' \citep[527f.]{hayek-1945} und kollektive Entscheidungen, wie zum Beispiel Planwirtschaft, desto besser.
Die Entscheidungsgewalt über ökonomische Handlungen sollte also so oft wie möglich bei den Individuen liegen.

Wenn man versucht, das auf das Schulsystem zu beziehen, könnte man sagen, dass die Idee eines Bildungskanons, einer Informationsvermittlung auf objektiver Basis, skeptisch betrachtet wird.
