\subsection[Illich]{Illich: Mythos Schule: Vorschläge einer alternativen Gesellschaft}

In ``Deschooling Society'' (1971) hinterfragt \citeauthor{Illich-1971} die Stellung und den gesellschaftlichen Einfluss von Schule.
Er eröffnet dabei neue Perspektiven auf die zentrale Frage unseres Kurses nach Formen menschlichen Zusammenlebens.
Sein Konzept beschränkt sich nicht darauf, das bestehende Schulsystem zu verändern, sondern strebt gleichzeitig gesellschaftliche Umwälzungen an.
\citeauthor{Illich-1971} begreift die Schule als ein Paradigma für die Natur von lebensbestimmenden Institutionen, fordert deren Abschaffung und plädiert für eine Gesellschaft, die sich für freie zwischenmenschliche Interaktion ausspricht.
Das Schulsystem erhalte nach \citeauthor{Illich-1971} die Illusion aufrecht, wir könnten Werte wie Bildung genau bestimmen, analysieren und schließlich messen.


\subsubsection{Kritik an der Institution Schule}

\epigraph{
	``Many students [...] intuitively know what the schools do for them. They school them to confuse process and substance.''
	\parencite[3]{Illich-1971}
}

Weil wir Prozess (z.B. Lehren) und Wert (z.B. Lernerfolg) verwechseln, nehmen wir fälschlicherweise an, das eine würde das andere zwangsläufig bedingen.
Führt mehr Lehren automatisch zu mehr Lernen?
Führt eine größere Polizeipräsenz automatisch zu mehr Sicherheit?
Würde dies zutreffen, könnte man Werte wie Sicherheit und Lernerfolg am Umfang des Prozesses (Polizeipräsenz, Schulanwesenheit) messen.
Den Mythos der messbaren Werte beziehen wir schließlich auch auf unsere ``imaginations, and, indeed, man himself '' \parencite[19]{Illich-1971}
Prozess und Wert beginnen einen allgemein verbindlichen Charakter zu entwickeln, weswegen das Wissen darüber an die nächste Generation weitergeben und deren Lernprozess somit ebenfalls kontrollieren werden.
Die Schule lehrt: Je mehr Schule, desto mehr Lernerfolg, weshalb wir zur Schule gehen.

Im Gegensatz dazu schreibt \citeauthor{Illich-1971}:
``Most learning is not the result of instruction: It is rather the result of unhampered participation in a meaningful setting.'' \parencite[18]{Illich-1971}.
Aus dieser Überlegung heraus errichtet er ein neues Konzept des Lernens in einer \emph{entschulten Gesellschaft}.


\paragraph{Lernen ohne Schule}

Für Bildung erachtet \citeauthor{Illich-1971} vier Aspekte als notwendig.
Jeder Mensch bedarf neben dem Kontakt zu anderen Gleichaltrigen auch Vorbilder, von denen er unbewusst lernen kann (Sprechen, Laufen, etc.).
Bei komplexeren Sachverhalten benötigt er professionelle Hilfe sowie den Zugang zu Lernobjekten.
Dies will \citeauthor{Illich-1971} über Netzwerke realisieren.
In einer dieser Netzwerke bekäme ein Kind, das Gitarre spielen lernen \emph, einen Zeitraum, in dem es mit anderen gleichaltrigen Kindern spielen kann, jemanden, der ihn bei Problemen und Schwierigkeiten unterstützen kann und als ein Art Vorbild funktioniert und als letztes eine Gitarre und alle anderen möglichen Werkzeuge, die dazu führen werden, dass das Kind erfolgreich den Lernprozess bewältigt.
Deutlich wird in diesem Beispiel, dass \citeauthor{Illich-1971} von einer hohen Selbstständigkeit des Individuums ausgeht.
In diesem Zusammenhang fordert er auch die Reduzierung der Komplexität unserer Umwelt, sodass sie wieder zugänglich für den Alltagsmenschen wird.
So könnten zum Beispiel Firmen subventioniert werden, die ihre Autos wieder leichter verständlich konzipieren, sodass jeder in der Lage ist, Reparaturen selbst durchzuführen.
So gäbe es auch einen praktischen Grund, der zum Lernen motivieren kann.


\paragraph{Illichs Entschulung der Gesellschaft als Lösungsansatz für die Organisation des menschlichen Zusammenlebens}

Durch sein Konzept eines selbstbestimmten und selbstorganisierten Lernens können viele der Anforderungen, welche die in der Exposition behandelten Texte an den Prozess des Lernens und an das menschliche Zusammenleben stellen, erfüllt und Widersprüche gelöst werden.
Illich beschreibt konkrete Vorschläge zur Umsetzung seiner ``educational revolution'', die als praktische Anwendung verschiedener theoretischer Ansätze verstanden werden können.

So würde der Konstruktivismus Illichs Auffassung des Lernbegriffes unterstützen.
Nach \citeauthor{siebert-2003} entscheide ``das psychische System, was es verarbeiten kann und will'', je nach dem es als viabel, d.h. bedeutend, von dem Individuum wahrgenommen werden \parencite[13]{siebert-2003}.
Jeder hat auch nach \citeauthor{Illich-1971} die Freiheit, eine Gelegenheit zum Lernen wahrzunehmen, oder sich dagegen zu entscheiden.

Damit sich dem Lernenden diese Wahlmöglichkeit eröffnet, benötigt er nach \citeauthor{Illich-1971} Zugang zu den vier oben beschriebenen Ressourcen.
Benner geht ebenfalls von gewissen Umständen, insbesondere Bezugspersonen, aus, die dem Zögling erst selbstbestimmtes Lernen ermöglichen.
Allerdings ``muss Erziehung stets dort an ihr Ende gekommen sein, wo pädagogische Fremdaufforderung zur Selbsttätigkeit in Selbstaufforderung übergehen kann.'' \parencite[91]{benner-2012}.
So könnte Benners Begriff der Bildsamkeit und Selbsttätigkeit Illichs Kritik an einem einheitlichen Schulsystem untermauern, welches pädagogischen Zwang auch ohne Bedarf fortführt.
Dennoch hält Benner den \emph{Pädagogen} und die \emph{Aufforderung zur Selbsttätigkeit} für notwendig; \citeauthor{Illich-1971} widerspräche dem.
Aber hier wird auch ein Defizit in Illichs Schreiben deutlich, da er sich nur mit dem gesunden Individuum auseinandersetzt.

Obwohl auch die Gruppierung von Gleichaltrigen nach \citeauthor{Illich-1971} abgeschafft werden soll, findet sein Lernen häufig in sozialen Kontexten statt
Ein Beispiel dafür sind seine \emph{Fertigkeitsbörsen}.
Dieser Vorgang lässt sich sehr gut mit dem Konzept des Hyperzyklus beschreiben, dessen Bedingung ist: ``wer etwas weiß oder kann, teilt es mit den anderen'' \parencite[123]{Zimpel2012}, was durch Illichs Vorschläge erfüllt wird.
Dadurch, dass nicht mehr wie vorher ``das Grundrecht auf Mitteilung des eigenen Wissens in das Privileg akademischer Freiheit [verkehrt wird], das nur den in einer Schule Beschäftigten verliehen wird'' \parencite[97]{Illich-1971}, ist jeder dazu befähigt, andere an seinem Wissen teilhaben zu lassen, während er wiederum von den Fähigkeiten anderer profitieren kann.

Der Ansatz des sozialen Lernens lässt sich auch auf \citeauthor{Dewey2010}  beziehen.
\citeauthor{Illich-1971} und \citeauthor{Dewey2010}  stellen sich beide gegen einen aktiven Lehrprozess und sehen Lernen eher als eine natürliche Eigenschaft des Menschen.
Auch \citeauthor{Dewey2010}  beschreibt seinen ungezwungenen Erfahrungsaustausch als wichtigste Eigenschaft einer fortschrittlichen Gesellschaft und leitet seinen Wert von der Menge an Interaktion innerhalb und mit einer anderen Gruppe \parencite[vgl.][89]{Dewey2010}.

Illichs Glaube an den Menschen außerhalb von sozialen Manipulationen ähnelt oft Rousseau, weshalb auch beide einen sehr skeptischen Blick mit fast verschwörungstheoretischen Charakter auf Gesellschaften als Perversionen von Menschsein werfen.
``I see, therefore, in love, hope, and charity the crowning of the proportional nature of creation in the full, old sense of that term'' (https://www.youtube.com/watch?v=vQLWAafp020, zuletzt geöffnet am 04.09.2014)
Rousseau schreibt: ``In der natürlichen Ordnung sind die Menschen alle einander gleich. Ihr gemeinsamer Beruf ist: Mensch zu sein.'' \parencite[50]{rousseau-1762}
Beide kritisieren die Identifikation durch Berufe und Rollenbilder, romantisieren den vormodernen Menschen und plädieren für eine unabhängige Erziehung.
Kooperations- und Verständigungsprobleme finden bei beiden wenig Beachtung.
Daraus ergibt sich ein optimistisches Bild einer Gesellschaft ohne Standards und jeglichen Verpflichtungen.


\paragraph{Gleichwertigkeit und Autonomie nach Illich}

Ein zentraler Punkt, den \citeauthor{Illich-1971} an Schule kritisiert, ist deren Recht zur Bewertung der Schüler durch Zertifikate, die auch in der gegenwärtigen Gesellschaft als einzig legitime Qualifikationen betrachtet werden.
Die Folgen dieser Macht über Bildungswege Einzelner sind weitreichend, was \citeauthor{Illich-1971} mit dem Satz ``School initiates young people into a world where everything can be measured, including their imaginations, and, indeed, man himself.''  \parencite[19]{Illich-1971}, veranschaulicht.
Schule produziert, indem sie alles als messbar ansieht, Ungleichheit.
Das impliziert die unterschiedliche Wertigkeit von Menschen und führt somit dazu, dass Schüler nicht nur als ungleich betrachtet werden, sondern auch sich selber als unterschiedlich wertvoll ansehen.
\citeauthor{Illich-1971} schlägt nun vor, das Messen von Fähigkeiten im Bezug auf den Zugang zu Bildungsmöglichkeiten ganz abzuschaffen.
So entsteht ein Menschenbild, dass von der Gleichwertigkeit aller ausgeht, während die beschriebenen Freiheiten und Wahlmöglichkeiten ein Maximum an persönlicher Autonomie ermöglichen.
