%!TEX root=./Emile.tex
\section[Kursreflexion]{Kooperation im Selbstexperiment: Reflexion über unsere Kursarbeit mit Git(Hub)}

Während unserer Kursarbeit haben unsere Kursleiter wohl oft mit Sorgenfalten in irritierte Gesichter geblickt.
Es hatte ja auch etwas heroisches an sich, die grundsätzlichsten Fragen der menschlichen Existenz im Sinne seiner Entwicklung und seiner soziologischen Gemeinschaft beantworten zu wollen.
Der eine oder andere Teilnehmende musste auf diesem Weg mehr als einmal seinen Blick auf die Welt ändern und rang besonders mit dem Begriff der Deontologie.
Da wir uns inhaltlich mit Fragen der menschlichen Kooperation (manchmal auch der tierischen) beschäftigt haben, lag es nahe, auch unsere Kursarbeit kooperativ zu organisieren.
Wie keiner vor uns haben wir versucht, diesen \emph{schizodisziplinären} Kurs und das Schreiben der Dokumentation auf der Open-Source-Plattform GitHub zu organisieren.
GitHub ist ein Programm, das ursprünglich zum Zweck der Koordination von Softwareentwicklung entstanden ist.
Vor Akademiebeginn arbeiteten wir als Vorbereitung für den Kurs bereits gemeinsam auf GitHub.com, der sozialen Komponente des Versionierungstools Git.
Später verwendeten wir ebenfalls Git, um die Dokumentation zu schreiben, in Kombination mit dem Open-Source-Texteditor Atom.
Diese drei Komponenten sollten unsere Arbeit so effizient wie möglich gestalten.

Jeder hatte zu jedem Zeitpunkt die Möglichkeit, auf alle geschrieben Texte zuzugreifen, diese zu kommentieren und selbst zu bearbeiten und zu ändern.
Dadurch entstand eine völlig neue Form der sozialen Interaktion, die in Bezug auf die Ausgangsfrage nach persönlicher Autonomie und inhärenter Gleichheit zu analysieren ist.
Hierfür lassen wir vier Autoren aus unserem Kurs sprechen und befragen sie postum, was sie zu diesem gewagten Unterfangen auf GitHub sagen würden.


\paragraph{Was Freinet zu unserer Zusammenarbeit auf GitHub sagen würde}

Während wir gerade fleißig die letzten Issues \emph{closen} und *\emph{merge conflicts} lösen, hantierte Célestin Freinet mit Druckerpressen und Karteikarten.
Lässt sich unsere gemeinsame Arbeit auf GitHub trotzdem mit seinen 100 Jahre alten, reformpädagogischen Ideen vereinen?

\begin{quote}
	``Arbeit in der Gruppe [aber] bedeutet nicht zwangsläufig, daß jeder die gleiche Arbeit verrichtet. Der Einzelne muß dabei im Gegenteil ein Maximum von seiner Persönlichkeit bewahren, aber im Dienst der Gemeinschaft stehen.''\\*
	\parencite[510]{Freinet-2000a}
\end{quote}

Dieses Problem löst Github durch allumfassende Berechtigungen für jeden TN.
Jeder arbeitet dadurch unabhängig und autonom in seiner Geschwindigkeit an selbstgewählten Themen.
Freinet unterstreicht die Wichtigkeit des Selberwählens aus gegebenen Alternativen \parencite[vgl.][495]{Freinet-2000a}.
Zugleich arbeiten wir kooperativ zusammen, denn durch die zeitgleiche, gegenseitige Kontrolle  wird das Ergebnis besser, als es in Einzelarbeit sein könnte.
Wir arbeiten damit im Sinne eines Kollektivs im Kurs.
Durch die virtuelle Vernetzung entsteht allerdings die Gefahr, dass die reale, emotionale Komponente verloren geht.
Für \citeauthor{Freinet-2000a} ist eine Gruppendynamik ähnlich der einer Familie sehr wichtig.
Ohne geographische Nähe ist diese nicht mehr gegeben.

Als Fazit lässt sich also sagen, dass \citeauthor{Freinet-2000a} unserem System wahrscheinlich nicht abgeneigt wäre, aber seine contra-sozialen Folgen stark kritisieren würde.


\paragraph{Illich zu Github}

\begin{quote}
	``Education for all means education by all.''\\*
	\parencite[17]{Illich-1971}.
\end{quote}

Würde \citeauthor{Illich-1971} unsere Verwendung von Github zum gemeinsamen Arbeiten an dieser Dokumentation und zur Vorbereitung unseres Kurses befürworten?

\citeauthor{Illich-1971} formuliert sogar schon in den Siebziger-Jahren ein sehr konkretes Bild von computergesteuerten Lernnetzwerken.
Seine ``learning webs'' sollten aber vor allem Menschen mit ähnlichen Interessen helfen zusammenzukommen.
Die eigentliche Organisation der Zusammenarbeit ist für \citeauthor{Illich-1971} von zweiter Bedeutung, weshalb Github eigentlich kein Diskussionsthema in Bezug auf sein Werk ``Deschooliing Society'' darstellt.
Als erster Kritikpunkt fällt hierbei natürlich die Vorauswahl der Teilnehmenden ins Auge.
GitHub würde für \citeauthor{Illich-1971} dann an Bedeutung gewinnen, wenn es sich hin zu einer allgemeinen Lernplattform ausweiten würde, zu der jeder Zugang hätte und in der jeder seine Ideen frei äußern kann.
Angewand auf seine ``learning webs'' müsste dann GitHub als Lernobjekt kategorisiert werden, so dass jedem Individuum der Zugang zu PCs mit mindestens Windows7 ermöglicht wird.

Aber der wahrscheinlicher wichtigere Aspekt in Bezug auf unsere Arbeit ist die Auseinandersetzung mit der Technik.
Wir werden dazu motiviert, uns mit neuen Themengebieten auseinanderzusetzen, da es einen direkten Bezug auf unsere Arbeit hat und wir es als bedeutend wahrnehmen.
Dadurch wird der Kontakt zu anderen Teilnehmern gefördert, mit denen wir zusammen versuchen, die Technik zu verstehen und zu den KLs, wodurch wir zu jedem Zeitpunkt die Möglichkeit haben , professionelle Hilfe zu bekommen. Diesen fällt eben keine pädagogisch belehrende Rolle zu und sie vermittlen uns nicht das eigentliche Wissen,
sondern sie nehmen eine beratende Stellung ein und helfen uns vor allem in organisatorischen, bzw. technischen Fragen weiter:
``He may invite the learner to participate in his own research'' \parencite[43]{Illich-1971}.
Jedoch stellen sich dennoch weitere Probleme in den Weg:
Schränkte man unser Projekt allein auf die Ausarbeitung auf Github ein, würde sich unser Kontakt ebenfalls wieder auf eine rein effizienzorientierte Ebene reduzieren: ``to reinforce the competitive nature of schools'' \parencite[35]{Illich-1971}, vor der \citeauthor{Illich-1971} warnt.


\paragraph{Habermas zu Github}

Würde Jürgen Habermas von unserer Zusammenarbeit auf GitHub hören, hätte er wahrscheinlich angesichts der Tatsache, dass er seine Texte bis heute auf einer Schreibmaschine schreibt, einige Bedenken.

Diese Arbeitsform ist im Kern allerdings von seinem Kommunikationsmodell gar nicht so weit entfernt, wie  wir im Laufe des Arbeitsprozesses feststellen konnten.

Zum einen ist das Ziel, dass durch das Einsetzen des Programms erreicht werden soll die Koordination der Kooperation - für Habermas der Grund, weshalb soziales Handeln überhaupt zu Stande kommt:
In dem Moment, in dem mehrere Menschen ein Ziel verfolgen ist eine Absprache nötig um einzelne Handlungen so zu koordinieren, dass man sich nicht in die Quere kommt.
Und genau diese Absprachen treffen wir immer wieder auf Github, indem wir versuchen unsere *Issues* und *Commits* hinzufügen und dabei immer den Anspruch haben den Anderen unsere Ziele verständlich zu machen.

Außerdem sind ständige Infragestellung von Gültigkeitsansprüchen essentiell für kommunikatives Handeln, das wir durch ständiges kommentieren, anzweifeln und ändern von Isssues und Dokumentationsbeiträgen in die Praxis umsetzen bis (möglichst) alle zufrieden sind.

Auch Habermas' Idee aus der Diskursethik, in der Sprechakte möglichst präzise formuliert werden sollen, lässt sich bei uns auf GitHub wiederfinden.
Indem wir versuchen, bei jedem Issue den Titel möglichst kurz und präzise mit einem passenden Operator zu formulieren, wenden wir seine Kommunikativa an, wie z.B. ``Klären, ob die Reflektion weiteres Feedback braucht''.
Werden dazu noch \emph{labels} vertagged, ist auch der Gültigkeitsbereich klar definiert (z.B. Dokumentation, Sozialwissenschaften).

Was Habermas an unserer Arbeitsform allerdings kritisieren könnte, ist die Instutionalisierung des Arbeitsprozesses.
Denn GitHub als Institution ersetzt in Teilen den direkten Austausch in intersubjektiven Lebenswelten durch ein systematischen Prozess, den Habermas in der modernen Gesellschaft stark problematisiert.


\paragraph{Fazit zu unserer Arbeit auf Git(Hub)}

Auch wenn die meisten der von uns behandelten Autoren vor der Zeit des Internet veröffentlicht hatten und ein Vergleich zu Github somit im Großteil der Fälle auf Spekulationen und Vermutungen beruht, lässt sich doch sagen, dass unsere Zusammenarbeit über Github in einigen Bereichen den Idealvorstellungen mehrerer Autoren bereits sehr nahekommt, besonders was die Kommmunikation und die fast uneingeschränkte Freiheit jedes einzelnen angeht, die für eine so bislang unbekannte Qualität der Kooperation und Abstimmung untereinander sorgt.
Allerdings ist auch Github natürlich nicht nicht das Blaue vom Himmel, so dass bei längerer Arbeit Fehler ans Tageslicht treten, wie zum Beispiel die akute Gefahr, dass durch die Arbeit auf GitHub die persönliche Kommunikation eingeschränkt und im schlimmsten Fall komplett durch die virtuelle ersetzt wird.
