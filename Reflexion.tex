%!TEX root=./Emile.tex
\section[Kursreflexion]{Kooperation im Selbstexperiment: Reflexion über unsere Kursarbeit mit Git(Hub)}

Während unserer Kursarbeit haben unsere Kursleitenden wohl oft mit Sorgenfalten in irritierte Gesichter geblickt.
Es hatte ja auch etwas heroisches an sich, die grundsätzlichsten Fragen der menschlichen Existenz im Sinne seiner Entwicklung und seiner soziologischen Gemeinschaft beantworten zu wollen.
Der eine oder andere Teilnehmende musste auf diesem Weg mehr als einmal seinen Blick auf die Welt ändern und rang besonders mit dem Begriff der Deontologie.
Da wir uns inhaltlich mit Fragen der menschlichen Kooperation (manchmal auch der tierischen) beschäftigt haben, lag es nahe, auch unsere Kursarbeit kooperativ zu organisieren.
Wie keiner vor uns haben wir versucht, diesen \emph{schizodisziplinären} Kurs und das Schreiben der Dokumentation mit Konventionen und Werkzeugen der Open-Source-Entwicklung zu organisieren.

Die drei Komponenten

\begin{enumerate}
	\item \emph{Git}, einem Versionskontroll- und Quellverwaltungsprogramm zur gemeinsamen Softwareentwicklung,
	\item \emph{Github.com}, einem kommerziellen Hoster von \emph{Git} und Anbieter eines darauf aufbauenden sozialen Netzwerks,
	\item \emph{Atom.io}, einem quelloffenen Software-Editor
\end{enumerate}

sollten unsere Arbeit so effizient wie möglich gestalten.

Vor Akademiebeginn arbeiteten wir als Vorbereitung für den Kurs bereits gemeinsam auf der Plattform GitHub.com, um dort Problemstellungen (``Issues'') zu bearbeiten und zu kommentieren.

Während des Kurses hatte jeder zu jedem Zeitpunkt die Möglichkeit, auf alle geschriebenen Texte zuzugreifen, diese zu kommentieren und selbst zu bearbeiten und zu ändern.
Dadurch entstand eine völlig neue Form der sozialen Interaktion, die in Bezug auf die Ausgangsfrage nach persönlicher Autonomie und inhärenter Gleichheit zu analysieren ist.
Hierfür lassen wir drei Autoren aus unserem Kurs sprechen und befragen sie postum, was sie zu diesem gewagten Unterfangen auf GitHub sagen würden.


\paragraph{Was Freinet zu unserer Zusammenarbeit auf GitHub sagen würde}

Während wir gerade fleißig die letzten Issues \emph{closen} und \emph{merge conflicts} lösen, hantierte Célestin Freinet noch mit Druckerpresse und Karteikärtchen, um Aufgaben zu verteilen.
Lässt sich unsere gemeinsame Arbeit auf GitHub trotzdem mit seinen 100 Jahre alten, reformpädagogischen Ideen vereinen?

\begin{quote}
	``Arbeit in der Gruppe [aber] bedeutet nicht zwangsläufig, daß jeder die gleiche Arbeit verrichtet. Der Einzelne muß dabei im Gegenteil ein Maximum von seiner Persönlichkeit bewahren, aber im Dienst der Gemeinschaft stehen.''\\*
	\textcite[510]{Freinet-2000a}
\end{quote}

Dieses Problem löst Github durch allumfassende Berechtigungen für jeden TN.
Jeder arbeitet dadurch unabhängig und autonom in seiner Geschwindigkeit an selbstgewählten Themen.
Freinet unterstreicht die Wichtigkeit der Selbstbestimmung aus gegebenen Alternativen \parencite[vgl.][495]{Freinet-2000a}.
Zugleich arbeiten wir kooperativ zusammen, denn durch die zeitgleiche, gegenseitige Kontrolle wird das Ergebnis besser, als es in Einzelarbeit sein könnte.
Wir arbeiten damit im Sinne eines Kollektivs im Kurs.
Durch die virtuelle Vernetzung entsteht allerdings die Gefahr, dass die reale, emotionale Komponente und der direkte Kontakt verloren geht.
Für \citeauthor{Freinet-2000a} ist eine Gruppendynamik ähnlich der einer Familie sehr wichtig.
Ohne räumliche Nähe ist diese nicht mehr gegeben.

Als Fazit lässt sich also sagen, dass \citeauthor{Freinet-2000a} unserem System wahrscheinlich nicht abgeneigt wäre, aber seine contra-sozialen Folgen stark kritisieren würde.


\paragraph{Was Illich zu unserer Zusammenarbeit auf GitHub sagen würde}

\begin{quote}
	``Education for all means education by all.''\\*
	\textcite[17]{Illich-1971}.
\end{quote}

Würde \citeauthor{Illich-1971} unsere Verwendung von Github zum gemeinsamen Arbeiten an dieser Dokumentation und zur Vorbereitung unseres Kurses befürworten?

\citeauthor{Illich-1971} formuliert sogar schon in den 1970er Jahren ein sehr konkretes Bild von computergesteuerten Lernnetzwerken.
Seine ``learning webs'' sollten vor allem Menschen mit ähnlichen Interessen helfen zusammenzukommen.
Als erster Kritikpunkt fällt hierbei natürlich die Vorauswahl der Teilnehmenden ins Auge.
GitHub würde für \citeauthor{Illich-1971} dann an Bedeutung gewinnen, wenn es sich zu einer allgemeinen Lernplattform ausweiten würde, zu der jeder Zugang hätte und in der jeder seine Ideen frei äußern könnte.
Angewandt auf seine ``learning webs'' müsste Git als Lernobjekt kategorisiert werden, das von allen tiefgreifend durchdrungen und verstanden wird.
Um die Lernobjekte in Illichs Sinne zu nutzen, wäre sicherlich ein größeres informatisches Verständnis notwendig, als wir es (einschließlich der Kursleitenden) derzeit haben.

Positiv zu erwähnen ist die Auseinandersetzung mit der Technik.
Wir werden dazu motiviert, uns mit neuen Themengebieten auseinanderzusetzen, da es einen direkten Bezug auf unsere Arbeit hat und wir es als bedeutend wahrnehmen.
Dadurch wird der Kontakt zu anderen Teilnehmenden gefördert, mit denen wir zusammen versuchen, die Technik zu verstehen und zu den Kursleitenden, wodurch wir zu jedem Zeitpunkt die Möglichkeit haben, Hilfe zu bekommen.
Beiden fällt keine belehrende Rolle zu und sie vermitteln uns nicht das eigentliche Wissen, sondern sie nehmen eine beratende Stellung ein und helfen uns vor allem in organisatorischen, bzw. technischen Fragen weiter:
``He may invite the learner to participate in his own research'' \parencite[43]{Illich-1971}.
Schränkte man unser Projekt jedoch allein auf die Ausarbeitung auf Github ein, würde sich unser Kontakt ebenfalls wieder auf eine rein effizienzorientierte Ebene reduzieren: ``to reinforce the competitive nature of schools'', vor der \textcite[35]{Illich-1971} warnt.


\paragraph{Was Habermas zu unserer Zusammenarbeit auf GitHub sagen würde}

Würde Jürgen Habermas von unserer Zusammenarbeit auf GitHub hören, hätte er wahrscheinlich angesichts der Tatsache, dass er seine Texte bis heute auf einer Schreibmaschine schreibt, einige Bedenken.

Diese Arbeitsform ist im Kern allerdings von seinem Kommunikationsmodell gar nicht so weit entfernt, wie wir im Laufe des Arbeitsprozesses feststellen konnten.

Zum einen ist das Ziel, dass durch das Einsetzen des Programms erreicht werden soll die Koordination der Kooperation --- für Habermas der Grund, weshalb soziales Handeln überhaupt zu Stande kommt:
In dem Moment, in dem mehrere Menschen ein Ziel verfolgen, ist eine Absprache nötig um die einzelne Handlungen zu koordinieren.
Genau diese Absprachen treffen wir immer wieder auf Github, indem wir \emph{Issues und Commits} hinzufügen und dabei immer den Anspruch haben, den Anderen unsere Ziele verständlich zu machen.

Außerdem ist die ständige Infragestellung von Gültigkeitsansprüchen essentiell für kommunikatives Handeln, das wir durch ständiges Kommentieren und Ändern von Isssues und Dokumentationsbeiträgen in die Praxis umsetzen bis (möglichst) alle zufrieden sind.

Auch Habermas' Idee einer der Diskursethik, in der Sprechakte möglichst präzise formuliert werden sollen, lässt sich bei uns auf GitHub wiederfinden.
Indem wir versuchen, bei jedem Issue den Titel möglichst kurz und präzise mit einem passenden Operator zu formulieren, wenden wir seine Kommunikativa an, wie z.B.\ ``Klären, ob die Reflektion weiteres Feedback braucht''.
Werden dazu noch \emph{Labels} vertagged, ist auch der Gültigkeitsbereich klar definiert.

Was Habermas an unserer Arbeitsform kritisieren könnte, ist die Institutionalisierung des Arbeitsprozesses.
Denn GitHub als Institution ersetzt in Teilen den direkten Austausch in intersubjektiven Lebenswelten durch ein systematischen Prozess, den Habermas in der modernen Gesellschaft stark problematisiert.


\paragraph{Fazit zu unserer Arbeit auf Git(Hub)}

Auch wenn die meisten der von uns behandelten Autoren vor der Zeit des Internets veröffentlicht hatten und ein Vergleich zu Github somit im Großteil der Fälle auf Spekulationen und Vermutungen beruht, lässt sich doch sagen, dass unsere Zusammenarbeit über Github in einigen Bereichen den Idealvorstellungen mehrerer Autoren bereits sehr nahekommt, besonders was die Kommunikation und die fast uneingeschränkte Freiheit jedes einzelnen angeht, die für eine so bislang unbekannte Qualität der Kooperation und Abstimmung untereinander sorgt.
Allerdings ist auch Github natürlich nicht der Weisheit letzter Schluss, sodass bei längerer Arbeit große Probleme ans Tageslicht treten, wie zum Beispiel die akute Gefahr, dass durch die Arbeit auf GitHub die persönliche Kommunikation eingeschränkt und im schlimmsten Fall komplett durch die virtuelle ersetzt wird.
