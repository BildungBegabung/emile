# Rousseaus idealistisches Gedankenexperiment

>"Man was born free and everywhere he lives in chain stores" (Jean-Jacques Rousseau / l`Internet)

Wie kann man Sozialwissenschaften und Pädagogik zusammendenken?
Wie persönliche Autonomie und inhärente Gleichheit verbinden?
Viele der Fragen, mit denen sich unserer Kurs beschäftigt, stellte sich Rousseau bereits im 18. Jahrhundert.
Auch wenn uns die Antworten, die er im Contrat Sociale und Émile gibt, nicht unbedingt befriedigen können, so helfen sie doch zu präzesieren, wovon unser Kurs und dieser Text handeln.


## Émile

Rousseaus Gesellschaftsverständnis baut hierbei stark auf dem Menschenbild auf, das er in seinem Erziehungsroman Émile entwickelt:
Noch vor der französischen Revolution oder der Unabhängigkeitserklärung der Vereinigten Staaten stellt er, Bürger eines absolutistischen Feudalsystems, die These auf, jeder Mensch sei frei, persönlich autonom und inhärent gleich.
Er schreibt: "Der Mensch wird frei geboren, und überall ist er in Banden." (Rousseau-1762, Contrat Social, S. 1)
Doch wenn der Mensch nicht natürlich erzogen wird, werden diese Grundeigenschaften überdeckt.
Als Beispiel, wie diese natürliche Erziehung aussehen soll, legt Rousseau die Erziehung seines fiktiven Zöglings Émile dar.

In der Kindheit wird der Junge, abgeschottet von der Gesellschaft, negativ erzogen.
Das heißt, das Kind lernt nichts über Moral, Religion, Ethik, ..., sondern es wird vor "Laster[n] und [...] Irrtümern bewahr[t]". (Rousseau-1762, Émile, S. 54)
Und zwar in einer Art extremen Laissez-faire.
Wenn das Kind etwa von ihm benötigte Gegenstände zerstört,
> "[b]eeilt euch nicht, ihm andere zu geben; laßt es empfinden, wie unangenehm es ist, sie nicht zu haben." (ebd., S. 54)

Bis zum 12. Lebensjahr soll das Kind so einfach leben, stark und gesund werden.
Ab dann beginnt man das Kind in Naturkunde zu unterrichten. (Vgl. ebd., S. 55)
In diesem Unterricht aber hat die Lehrerin/Erzieherin nur die Aufgabe, Lernsituationen zu schaffen.
Sie soll nichts erklären, sondern den Zögling allein herausfinden und verstehen lassen. (Vgl. ebd., S. 56)

Erst wenn das Kind erwachsen wird, beginnt man mit der positiven Erziehung:
Das Kind wird nicht mehr als Zögling, sondern als Freund behandelt und moralisch, gesellschaftlich und religiös unterwiesen.
Es ist nun in der Lage, eigene fundierte Entscheidungen, wie die Wahl seiner Religion, zu treffen. (Vgl. ebd., S. 60f)

Bald darauf kommt die Zeit, dass Émile in die Gesellschaft eingeführt werden muss, da der Mensch nicht für immer allein bleiben kann (Vgl. ebd., S. 61.)
Der jetzt Erwachsene hat die Fähigkeit erworben, auch in der Gesellschaft seine Aufgabe zu erfüllen, Mensch zu bleiben.
>"In der natürlichen Ordnung sind die Menschen alle einander gleich. Ihr gemeinsamer Beruf ist: Mensch zu sein."
(ebd., S.50)

Was passiert aber mit derjenigen, die nicht natürlich erzogen wird?
Ihre bürgerliche Erziehung zerteilt sie in verschiedene Rollen und Identitäten, deren verschiedene Pflichten nicht miteinander zuvereinbaren sind.
Sie ist z.B. als Abgeordnete dem Fraktionszwang unterworfen, für etwa eine Steuer zustimmen, und als Angehörige eines Berufsstandes muss sie dagegen sein.
Oder sie muss als Teilnehmerin der KüMu zu einer Probe und als Kursteilnehmerin Protokoll schreiben ...
Dadurch, dass sie sich über verschiedene Aufgaben definiert, kann es also leicht zu inneren Rollenkonflikten kommen.


## Le Contrat Social

Während die bürgerlich Erzogene also in jeder ihrer Rollen Partikularinteressen hat, kann ein natürlicher Mensch, der nur Mensch ist, das allgemeine menschliche Interesse des volonté générale erkennen.
Der volonté générale, der Gemeinwille, geht davon aus, dass es ein vom Individuum losgelöstes abstraktes Interesse der Menschheit gibt.
Und nur dieses Interesse kann in einem staatlichen System absolute Freiheit und Gleichheit garantieren.
Im Urzustand der Menschheit hingegen war die Frage der Vereinbarbarkeit von Freiheit und Gleichheit völlig nebensächlich, da es keine institutionelle Kooperation gab.
Das heißt, dass der Weg der Kooperation durch Traditionen, Gesetze oder anderes vorgeschrieben ist, beziehungsweise wir nicht als natürlich Menschen, sondern z.B. als DSA-Teilnehmerinnen zusammenarbeiten.
Für Rousseau ist der Urzustand zwar das absolute Optimum, jedoch ist dieser unwiederbringlich verloren.
Deshalb schlägt Rousseau den Weg über den volonté générale als Lösung vor.
Nur er kann und muss staatliche Gewalt rechtfertigen.

In einem so gelenkten Staat herrscht Gleichheit, da jedes Individuum und die Gesellschaft im Ganzen das Bestmögliche erhalten (Vgl. Rousseau-1762, Contrat Social, S. 7);
und Freiheit, weil jedes Mitglied sich freiwillig der Gemeinschaft unterordnet.
Im Zweifelsfall muss der Mensch jedoch dazu gezwungen werden, diese Freiheit wahrzunehmen.
Dabei gilt für ihn "Each man [woman] in giving himself [herself] to everyone gives himself to no-one" (ebd., S. 7)
Er bleibt also vollkommen autonom.

Rousseau hat damit innerhalb seines Gedankenkonstrukts eine Lösung des Vereinbarkeitsproblems gefunden
Seine Intention lag dabei darin, ein normatives Gebilde zu errichten, nicht aber darin, eine Verfassung zu schreiben.
Dennoch hat er als "1. Aufklärer", der eine göttliche Ordnung bestritt, großen Anteil an den freiheitlich demokratischen Entwicklungen der Folgezeit.

Rousseaus romantisch verklärten Ansichten bereiten zwar manche Verständnisprobleme, doch wir müssen festhalten, dass seine antimodernen Ideen faszinieren und viele Anregungen für die weitere Arbeit mitgeben.
